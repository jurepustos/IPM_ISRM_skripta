\documentclass[11pt, a4paper]{article}
\usepackage{mathtools}
\usepackage{caption}
\usepackage{float}
\usepackage[slovene]{babel}
\usepackage{amssymb}
\usepackage{amsthm}
\usepackage{enumitem}
\usepackage{mathrsfs}



\hyphenpenalty=10000

\begin{document}
    \newtheorem{theorem}{Izrek}[subsection]
    \newtheorem{definition}[theorem]{Definicija}
    \newtheorem{corollary}[theorem]{Posledica}
    \newtheorem{lemma}[theorem]{Lema}
    \newtheorem{proposition}[theorem]{Trditev}
    \newtheorem{example}[theorem]{Zgled}

    \newtheorem*{remark}{Opomba}


    \title{Izbrana poglavja iz matematike}
    \author{Napisal Jure Pustoslemšek po zapiskih predavanj prof. dr. Petarja Pavešiča}
    \date{Junij 2020}
    \maketitle

    \section{Abstraktna algebra: kolobarji in obsegi}

    \begin{definition}[Kolobar]
        Kolobar je množica, kateri smo priredili notranji operaciji seštevanja in množenja, ki zadostujeta spodnjim kriterijem. Eksplicitno ga lahko zapišemo kot \((K,+,\cdot)\).
        \par
        Pri \underline{seštevanju} velja \underline{komutativnost} in \underline{asociativnost} obstaja ničla 0 in za vsak \(a \in K\) obstaja nasprotni element \((-a) \in K\), torej lahko vedno odštevamo.
        \[\forall a,b \in K: a + b = b + a\]
        \[\forall a,b,c \in K: (a + b) + c = a + (b + c)\]
        \[\forall a \in K \exists (-a) \in K: a + (-a) = a - a = 0\]
        
        \par
        Pri \underline{množenju} nimamo dodatnih zahtev, wazen uglašenosti s seštevanjem - \underline{distibutivnost}.
        \[\forall a,b,c \in K: a \cdot (b+c) = (a \cdot b) + (a \cdot c)\]

        Rečeno drugače, kolobar je grupa za seštevanje in zaprta za množenje.
    \end{definition}

    Če ima množenje kakšno dodatno lastnost, to lastnost običajno izpostavimo\
    \begin{center}
        \begin{tabular}{ c c c }
            Množenje je asociativno & \(\rightarrow\) & asociativni kolobar \\
            Množenje je komutativno & \(\rightarrow\) & komutativni kolobar \\
            Množenje ima enoto & \(\rightarrow\) & kolobar z enoto \\
            Vsak \(a \neq 0 \in K\) ima inverz \(a^{-1}\) & \(\rightarrow\) & kolobar z deljenjem
        \end{tabular}        
    \end{center}

    Literatura velikokrat v definiciji kolobarja zahteva tudi asociativnost in obstoj enote, zato bomo v nadaljevanju privzeli, da z izrazom "kolobar" mislimo na asociativen kolobar z enoto, komutativnost in deljenje pa bomo izrecno navedli.
    
    \begin{definition}[Obseg]
        Kolobarju, v katerem je množenje asociativno, komutativno in ima enoto ter ima operacijo deljenja, rečemo \textbf{obseg}.
    \end{definition}

    \subsection{Kolobarji}

    \begin{example}
        Če dani kolobar nima enote, mu jo lahko dodamo: \\
        \(K\) kolobar brez enote \\
        Če dodamo 1, smo prisiljeni dodati tudi \(-1,2,-2,3,-3,...\) \\
        Rešitev: na \(\mathbb{Z} \times K \) vpeljemo:
        \[
            (n,a)+(m,b) := (n+m,a+b)
        \]
        \[
            (n,a) \cdot (m,b) := (nm, nb + ma + ab)
        \]

        Ničla je \((0,0)\), nasprotni element je \(-(m,a) = (-m,-a)\), enota je \((1,0)\). Če je \(K\) komutativen oz. asociativen, je to tudi \(\mathbb{Z} \times K\).
    \end{example}

    Kaj pa, če imamo kolobar, v katerem nekateri elementi nimajo inverza, ampak bi jih želeli dodati? Poskusimo to storiti na takšen način, kot smo v osnovni šoli definirali racionalna števila s celimi števili, in sicer z uvedbo ulomkov. 

    \par
    Naj bo \(K\) kolobar z enoto. Za \(a,b \in K, b \neq 0\) vpeljemo simbole \(\frac{a}{b}\), s katerimi računamo 
    \[\frac{a}{b} + \frac{c}{d} := \frac{ad + bc}{bd}\] 
    \[\frac{a}{b} \cdot \frac{c}{d} := \frac{ac}{bd}\]

    Pojavi se težava: lahko se zgodi, da je \(bd = 0\), čeprav \(b \neq 0\) in \(d \neq 0\). Tega pri številih nismo vajeni vendar:
    \begin{example}[Matrike]
        \[
            \begin{bmatrix}
                0 & 1 \\
                0 & 0 \\
            \end{bmatrix}
            \cdot
            \begin{bmatrix}
                0 & 1 \\
                0 & 0 \\
            \end{bmatrix}
            =
            \begin{bmatrix}
                0 & 0 \\
                0 & 0 \\
            \end{bmatrix}
        \]
    \end{example}

    \begin{example}[\(\mathbb{Z}_{12}\): ostanki po modulu 12]
        \[2 \cdot 6 = 0\]
        \[3 \cdot 8 = 0\]
    \end{example}
    
    \begin{definition}[Delitelj niča]
        V kolobarju \(K\) je \(0 \neq a \in K\) \textbf{delitelj niča}, če obstaja tak \(b \neq 0\), da je \(a \cdot b = 0\).
    \end{definition}

    \begin{proposition}
        Delitelj niča nima inverza.
    \end{proposition}

    \begin{proof}
        Če za \(a \in K\) obstajata takšna neničelna \(b,c \in K\), da velja
        \begin{center}
            \(a \cdot b = 0\) in \(c \cdot a = 1\)
        \end{center}
        dobimo protislovje
        \[b = 1 \cdot b = (c\cdot a) \cdot b = c \cdot (a \cdot b) = c \cdot 0 = 0\]
    \end{proof}

    \begin{definition}[Celi kolobar]
        \textbf{Celi kolobar} je komutativni kolobar, v katerem ni deliteljev niča. Ekvivalentno, v celem kolobarju iz \(a \cdot b = 0\) sledi \(a = 0\) ali \(b = 0\).
    \end{definition}

    Naj bo \(K\) celi kolobar. Tvorimo \textbf{kolobar ulomkov} \(\overline{K}\): elementi so ulomki \(\frac{a}{b}\), vendar \(\frac{a}{b} \equiv \frac{c}{d}\), če je \(ad = bc\) (ulomke lahko krajšamo). Definiramo operaciji kot prej:
    \[\frac{a}{b} + \frac{c}{d} = \frac{ad + bc}{bd}\]
    \[\frac{a}{b} \cdot \frac{c}{d} = \frac{ac}{bd}\]

    Preverimo lahko, da je \(\overline{K}\) obseg.
    \par
    Zakaj moramo enačiti sorazmerne ulomke?
    \par
    Zaradi definicije operacij: \(\frac{a}{b} - \frac{ka}{kb} = \frac{kab - kab}{kb^2} = 0\), torej mora veljati \(\frac{ka}{kb} \equiv \frac{a}{b}\).

    \begin{remark}
        Z nekaj truda lahko vpeljemo ulomke tudi pri nekomutativnih kolobarjih in kolobarjih z deliterlji niča. Takrat dobimo kolobarje ulomkov, ki niso obsegi.
    \end{remark}

    Delitelj niča ne more biti obrnljiv, obrnljiv element pa ni delitelj niča. Ali je lahko element kolobarja niti obrnljiv niti delitelj niča?
    \begin{example}
        \(\mathbb{Z}\) nima deliteljev niča, obrnljiva pa sta le \(1\) in \(-1\).
    \end{example}

    \begin{example}
        \(\mathbb{Z}_{12}\)
        \begin{itemize}
            \item Delitelji niča: \(0,2,3,4,6,8,9,10\)
            \item Obrnljivi: \(1,5,7,11\)
        \end{itemize}
    \end{example}

    V končnih kolobarjih ni drugih možnosti, kar pove naslednji znameniti izrek.

    \begin{theorem}[Wedderburnov izrek]
        Končen kolobar brez deliteljev niča je obseg.
    \end{theorem}

    \begin{proof}
        Naj bo \(K\) končen kolobar brez deliteljev niča. Dokazati moramo, da so vsi elementi \(K - \{0\}\) obrnljivi.
        \par
        Za poljuben \(a \in K - \{0\}\) definiramo funkcijo
        \[l_a: K \rightarrow K\] 
        \[l_a(x) := a \cdot x\]
        \par
        Recimo, da za neka \(x,y \in K\) velja \(l_a(x) = l_a(y)\). Potem je \(ax = ay\), torej \(a(x-y) = 0\). Ker \(K\) nima deliteljev niča, sledi \(x = y\). Torej je \(l_a\) injektivna in ker slika iz \(K\) v \(K\), tudi bijektivna.
        \par
        Ker je \(l_a(K) = K\), je \(l_a(b) = a \cdot b = 1\) za nek \(b \in K\). Ponovimo razmislek za \(d_a (x):= x \cdot a\) in dobimo, da je \(d_a(c) = a \cdot c = 1\) za nek \(c \in K\). Iz \(c = c \cdot 1 = c \cdot (a \cdot b) = (c \cdot a) \cdot b = 1 \cdot b = b\) sledi \(c = b = a^{-1}\). \(K\) je torej kolobar z deljenjem. Če je \(K\) komutativen, je obseg.
        \par
        Dokaz komutativnosti zahteva nekaj dodatne teorije grup, zato ga ne bomo natančno navedli. Ideja je, da je center \(Z(K)\) (tj. elementi \(K\), ki komutirajo z vsemi elementi \(K\)) obseg, celoten \(K\) pa je vektorski prostor nad \(Z(K)\). Iz primerjave multiplikativnih grup lahko izpeljemo, da je \(K\) 1-razsežen vektorski prostor nad \(Z(K)\), torej \(K = Z(K)\).
    \end{proof}

    \begin{corollary}
        \(\mathbb{Z}_n\) je obseg \(\Leftrightarrow\) \(n\) je praštevilo.
    \end{corollary}

    \begin{definition}[Karakteristika kolobarja]
        \textbf{Karakteristika} kolobarja \(K\) je najmanjši \(n \in \mathbb{N}\), za katerega velja, da je \(n \cdot a = a+...+a=0\). Zapišemo jo kot \(char(K)=n\) ali \(char K = n\). Če tak \(n\) ne obstaja, potem pišemo \(char(K) = 0\).
    \end{definition}

    \begin{proposition}
        Naj bo \(K\) kolobar.
        \begin{enumerate}[label=\alph*)]
            \item Če \(1 \in K\), potem je \(char(K) = \) red enote \(= min\ n\), da je \(n \cdot 1 = 0\).
            \item Če \(K\) nima deliteljev niča, potem je \(char(K)\) praštevilo ali \(0\). 
        \end{enumerate}
    \end{proposition}

    \begin{proof}[Dokaz \emph{(a)}]
        Naj bo \(n\) red enote, torej je \(n \cdot 1 = 0\). Potem za vsak \(a \in K\) velja:
        \[n \cdot a = (n \cdot 1) \cdot a = 0 \cdot a = 0\]
        Iz tega sledi, da je \(char K \le n\). Ker je red \(1\) enak \(n\), je \(char K \ge n\), torej \(char K = n\).
    \end{proof}

    \begin{proof}[Dokaz \emph{(b)}]
            Denimo, da je \(char K = k \cdot l\) za \(k,l > 1\). Potem je \(0 = k \cdot l \cdot 1 = k \cdot l\). Ker v K ni deliteljev niča, je \(k \cdot 1 = 0\) ali \(l \cdot 1 = 0\), zato je po \((a)\) \(char K < k \cdot l\). Protislovje.
    \end{proof}

    \begin{definition}
        Naj bosta \(K,L\) kolobarja.
        \par
        \(f\): \(K \rightarrow L\) je \textbf{homomorfizem kolobarjev}, če velja:
        \begin{center}
            \(f(a + b) = f(a) + f(b)\)
        \end{center}
        \begin{center}
            \(f(a \cdot b) = f(a) \cdot f(b)\)
        \end{center}
        za poljubna \(a,b \in K\).
    \end{definition}

    Bijektivnemu homomorfizmu rečemo \textbf{izomorfizem}.
    \par
    Poglejmo nekaj primerov homomorfizmov:

    \begin{example}
        \(\mathbb{Z} \xrightarrow{mod\ n} \mathbb{Z}_n\) je homomorfizem (dokaz ponovi korake dokaza, da je \(\mathbb{Z}_n\) kolobar).
    \end{example}

    \begin{example}
        Konjugiranje \(\mathbb{C} \to \mathbb{C}\), \(z \mapsto \overline{z}\) je izomorfizem kolobarjev.
    \end{example}

    \begin{example}
        Za poljuben \(a \in \mathbb{R}\) definiramo \(f_a\): \(\mathbb{Z}[x] \rightarrow \mathbb{R}\) s predpisom \(f_a(p) = p(a)\)
        \[f_a(p + q) = (p + q)(a) = p(a) + q(a) = f_a(p) + f_a(q)\]
        \[f_a(p + q) = (p \cdot q)(a) = p(a) \cdot q(a) = f_a(p) \cdot f_a(q)\]
    \end{example}

    \begin{example}
        \(a \mapsto 
        \begin{bmatrix} 
            a & 0 \\ 
            0 & 0 
        \end{bmatrix}\) 
        je homomorfizem \(\mathbb{R} \rightarrow M_2(\mathbb{R})\).
    \end{example}

    \begin{example}
        Splošneje \(f_a\): \(\mathscr{C}(\mathbb{R}, \mathbb{R}) \rightarrow \mathbb{R}\) s predpisom \(f_a(g) := g(a)\) je tudi homomorfizem kolobarjev.
    \end{example}

    \begin{example}
        \(a \mapsto 
            \begin{bmatrix} 
                0 & a \\ 
                0 & 0 
            \end{bmatrix}\) 
        \underline{ni} homomorfizem \(\mathbb{R} \rightarrow M_2(\mathbb{R})\) (ohranja seštevanje, ne pa množenja).
    \end{example}

    \begin{example}
        Preslikava \(\mathbb{C} \rightarrow M_2(\mathbb{R})\) s predpisom \(a + bi \mapsto 
            \begin{bmatrix}
                a & b \\
                -b & a
            \end{bmatrix}\)
        je homomorfizem.
    \end{example}

    \begin{example}
        \(a \mapsto a^p\) je homomorfizem \(\mathbb{Z}_p \rightarrow \mathbb{Z}_p\) (\(p\) praštevilo).
    \end{example}

    \begin{proposition}
        Za homomorfizem \(f\): \(K \rightarrow L\) velja:
        \begin{enumerate}[label=\alph*)]
            \item \(f(0) = 0\)
            \item \(f(-a) = -f(a)\)
        \end{enumerate}
    \end{proposition}
    
    \begin{proof}[Dokaz \emph{(a)}]
        \(f(0) = f(0 + 0) = f(0) + f(0)\)
    \end{proof}

    \begin{proof}[Dokaz \emph{(b)}]
        \(0 = f(0) = f(a+(-a)) = f(a) + f(-a)\)
    \end{proof}

    \begin{definition}[Unitalen homomorfizem]
        Homomorfizem, za katerega velja \(f(1) = 1\), je \textbf{unitalen}.
    \end{definition}

    \begin{proposition}
        Če je \(f\) unitalen homomorfizem in \(a\) obrnljiv, je \(f(a)\) obrnljiv.
    \end{proposition}

    \begin{proof}
        Naj bo \(f: K \to L\) unitalen homomorfizem kolobarjev in \(a \in K\) obrnljiv. Potem je
        \[1 = f(1) = f(a \cdot a^{-1}) = f(a) \cdot f(a^{-1})\]
        Potem je \(f(a^{-1})\) desni inverz za \(f(a)\). Po podobnem razmisleku vidimo, da je \(f(a^{-1})\) tudi levi inverz za \(f(a)\). Torej je \(f(a^{-1}) = f(a)^{-1}\).
    \end{proof}

    \begin{proposition}
        Naj bo \(f\): \(K \to L\) homomorfizem kolobarjev. Potem je \(Im f\) podkolobar v \(L\).
    \end{proposition}

    \begin{proof}
        Vzamemo poljubna \(x,y \in Im f\). Potem obstajata takšna \(a,b \in K\), da je \(f(a) = x\) in \(f(b) = y\) in zato velja
        \[x + y = f(a) + f(b) = f(a + b) \in Im f\]
        \[x \cdot y = f(a) \cdot f(b) = f(a \cdot b) \in Im f\]
        Torej je \(Im f\) podkolobar v \(L\).
    \end{proof}

    \begin{proposition}
        Naj bo \(f: K \to L\) homomorfizem kolobarjev. Potem je \(Ker f\) podkolobar v \(K\).
    \end{proposition}

    \begin{proof}
        Vzamemo poljubna \(x,y \in Ker f\). Potem je \(f(x + y) = f(x) + f(y) = 0 + 0 = 0\), torej je \(x + y \in Ker f\). Podobno je \(f(x \cdot y) = f(x) \cdot f(y) = 0 \cdot 0 = 0\), torej je \(x \cdot y \in Ker f\). 
    \end{proof}


    \begin{definition}[Ideal kolobarja]
        Podkolobar \(I \leq K\) je \textbf{ideal} v \(K\), če za vsak \(a \in K, x \in I\) velja \(a \cdot x \in I\) (levi ideal) in \(x \cdot a \in I\) (desni ideal). Označimo ga z \(I \triangleleft K\).
    \end{definition}

    \begin{example}
        \(\{0\} \triangleleft K\) in \(K \triangleleft K\) sta "neprava" ideala.
    \end{example}

    \begin{example}
        \(n \mathbb{Z} \triangleleft \mathbb{Z}\)
    \end{example}

    \begin{example}
        \(\{a_1 x + a_2 x^2 + a_3 x^3 + ... + a_n x^n\ |\ a \in \mathbb{R}\} \triangleleft \mathbb{R}[x]\) so polinomi, za katere je \(p(0) = 0\). Splošneje, za \(a \in K\) je \(\{p \in K[x]\ |\ p(a) = 0\} \triangleleft K[x]\).
    \end{example}

    \begin{example}
        \(\{
            \begin{bmatrix}
                x & 0 \\
                y & 0 \\
            \end{bmatrix}\
            |\ x,y \in \mathbb{R}
        \} \subseteq M_2(\mathbb{R})\) je podkolobar.

        \[
            \begin{bmatrix}
                x & 0 \\
                y & 0 \\
            \end{bmatrix}
            \pm
            \begin{bmatrix}
                z & 0 \\
                w & 0 \\
            \end{bmatrix}
            =
            \begin{bmatrix}
                x \pm z & 0 \\
                y \pm w & 0 \\
            \end{bmatrix}
        \]

        \[
            \begin{bmatrix}
                x & 0 \\
                y & 0 \\
            \end{bmatrix}
            \cdot
            \begin{bmatrix}
                z & 0 \\
                w & 0 \\
            \end{bmatrix}
            =
            \begin{bmatrix}
                xz & 0 \\
                yw & 0 \\
            \end{bmatrix}
        \]
        Poleg tega je
        \[
            \begin{bmatrix}
                a & b \\
                c & d \\
            \end{bmatrix}
            \cdot
            \begin{bmatrix}
                x & 0 \\
                y & 0 \\
            \end{bmatrix}
            =
            \begin{bmatrix}
                ax + by & 0 \\
                cx + dy & 0 \\
            \end{bmatrix}
        \]
        \[
            \begin{bmatrix}
                x & 0 \\
                y & 0 \\
            \end{bmatrix}
            \cdot
            \begin{bmatrix}
                a & b \\
                c & d \\
            \end{bmatrix}
            =
            \begin{bmatrix}
                ax & bx \\
                ay & by \\
            \end{bmatrix}
        \]
        Podana množica je levi ideal. 
    \end{example}

    \begin{proposition}
        Naj bo \(f\): \(K \to L\) homomorfizem kolobarjev. Potem je \(Ker f\) (dvostranski) ideal v \(K\).
    \end{proposition}

    \begin{proof}
        Vzemimo \(x \in Ker f\) in \(a \in K\). \(f(x) \cdot f(a) = f(x) \cdot f(a) = 0 \cdot f(a) = 0\), torej je \(x \cdot a \in Ker f\). Analogno, \(f(a \cdot x) = f(a) \cdot f(x) = f(a) \cdot 0 = 0\), torej je tudi \(a \cdot x \in Ker f\).
    \end{proof}

    Za \(x \in K\) je \(K \cdot x = \{a \cdot x\ |\ a \in K\}\) levi ideal v \(K\), \(x \cdot K\) pa desni. Dvostranski ideal, ki ga generira \(x\), pa ni \(K \cdot x \cdot K\), ker ni zaprt za seštevanje. Potrebno je vzeti vse možne vsote izrazov \(a \cdot x \cdot b\).

    \begin{definition}[Glavni ideal]
        Ideal, ki ga generira en sam element, imenujemo \textbf{glavni ideal} in ga zapišemo z \((x)\), kjer je \(x \in K\).
    \end{definition}

    \begin{proposition}
        V \(\mathbb{Z}\) so vsi ideali glavni.
    \end{proposition}

    \begin{proof}
        Naj bo \(I \triangleleft \mathbb{Z}\). Če je \(I = \{0\}\), potem je \(I = (0)\). Recimo, da \(I \neq (0)\). Potem obstaja vsaj eno pozitivno število v \(I\). Vzemimo najmanjše pozitivno število \(a \in I\). Pokazati moramo, da \(a\) deli vse elemente \(I\).
        \par
        Poljuben \(b \in I\) je po izreku o deljenju enak \(b = k \cdot a + r\) za natanko določena \(k \in \mathbb{Z}\) in \(0 \le r < a\). Potem je \(r = b - k \cdot a \in I\). Ker je \(a\) najmanjše pozitivno število v \(I\), je \(r = 0\), torej je \(b = k \cdot a\). \(I\) je torej generiran z \(a\).
    \end{proof}

    \begin{definition}[Glavnoidealski kolobar]
        Kolobar, v katerem so vsi ideali glavni, je \textbf{glavnoidealski}.
    \end{definition}


    \begin{proposition}
        Naj bo \(K\) obseg. Vsi ideali v \(K[x]\) so glavni.
    \end{proposition}

    \begin{proof}
        Vzemimo \(I \triangleleft K\). Če je \(I = {0}\), je \(I = (0)\). Privzemimo, da je \(I \neq (0)\): Potem obstaja neničelni nekonstantni polinom najmanjše stopnje \(p \in K[x]\), za katerega lahko privzamemo, da je moničen. Pokažimo, da \(p\) deli vse elemente \(I\).
        \par
        Vzemimo poljuben \(q \in I\). Po izreku o deljenju je \(q(x) = k(x) p(x) + r(x)\) za natanko določena \(k \in K[x]\) in \(r \in K[x]\), ki je strogo manjše stopnje kot \(p\). Vidimo, da je \(r(x) = q(x) - k(x) p(x) \in I\). Ker pa je \(p\) polinom najmanjše stopnje med vsemi v \(I\), mora biti \(r(x) = 0\). To pa pomeni, da je \(q(x) = k(x)p(x)\). Ideal \(I\) je torej generiran z \(p\). Sledi, da so vsi ideali v \(K[x]\) glavni.
    \end{proof}


    Naj bo \(I \triangleleft K\) dvostranski ideal. Na \(K\) vpeljemo relacijo \(\sim\):
    \[a \sim b \Longleftrightarrow a - b \in I\]

    \begin{proposition}
        \(\sim\) je ekvivalenčna relacija.
    \end{proposition}

    \begin{proof}
        \(\)\par
        \(a - a = 0 \in I\), torej je \(a \sim a\).
        \par
        Če \(a - b \in I\), je tudi \(b - a \in I\), torej \(a \sim b \implies b \sim a\).
        \par
        Če \(a - b \in I\) in \(b - c \in I\), je tudi \(a - c \in I\). Torej je \(a \sim b, b\ \sim c \implies a \sim c\).
    \end{proof}

    \begin{definition}[Kvocientni kolobar]
        Množico ekvivalenčnih razredov relacije \(\sim\) označimo z \(K / I\) in jo imenujemo \textbf{kvocientni kolobar}. 
    \end{definition}

    Ekvivalenčni razred elementa \(a \in K\) označimo \([a]\) ali \(a + I\) (oznaka je smiselna, ker je \([a] = \{a + x\ |\ x \in I\}) \equiv a + I\).

    \par
    Elemente \(K / I\) naravno seštevamo in množimo:
    \[(a+I) + (b+I) := (a+b) + I\]
    \[(a+I) \cdot (b+I) := (a \cdot b) + I)\]

    Da bosta operaciji dobro definirani, moramo preveriti, sta neodvisni od izbire predstavnikov ekvivalečnih razredov. Najprej preverimo operacijo seštevanja:
    \par
    Če \(a' + I = a + I\) in \(b' + I = b + I\), je \(a - a' \in I\) in \(b - b' \in I\).
    \[(a - a') + (b - b') = (a + b) - (a' + b') = (a + b) + I = (a' + b') + I\]
    
    Preverimo še za množenje:
    \begin{center}
        \(a' = a + x\) in \(b' = b + y\) za \(x,y \in I\).
    \end{center}
    
    \[a' \cdot b' = a \cdot b + ay + xb + xy\]
    \[(a' \cdot b') + I = (a \cdot b) + I\]

    Zdaj vemo, da sta operaciji dobro definirani. Sedaj preverimo, da je \(K / I\) res kolobar.

    \begin{proposition}
        Naj bo \(K\) kolobar in \(I \triangleleft K\). Potem je \(K / I\) tudi kolobar.
    \end{proposition}

    \begin{proof}
        Vzemimo poljubne \(a,b,c \in K / I\). Najprej preverimo, da je grupa za seštevanje:
        \begin{center}
            asociativnost: \(((a+b) + c) + I = (a+I) + (b+I) + (c+I) = (a+I) + ((b+c) + I) = (a + (b+c)) + I\) \\
            komutativnost: \((a+b) + I = (a+I) + (b+I) = (b+I) + (a+I) = (b+a) + I\) \\
            negativni element: \((a+b) + I = 0 + I \implies a + I = -b + I\)
        \end{center} 

        Preverimo še asociativnost in obstoj enote za množenje:
        \begin{center}
            asociativnost: \(((a \cdot b) \cdot c) + I = (a+I) \cdot (b+I) \cdot (c+I) = (a+I) \cdot ((b \cdot c) + I) = (a \cdot (b \cdot c)) + I\) \\
            komutativnost: \((a \cdot b) + I = (a+I) \cdot (b+I) = (b+I) \cdot (a+I) = (b \cdot a) + I\) \\
        \end{center}
    \end{proof}

    \begin{theorem}[Izrek o izomorfizmu]
        Naj bo \(f\): \(K \to L\) homomorfizem kolobarjev. Potem je \(Ker f \triangleleft K\) in imamo naravni izomorfizem:
        \begin{center}
            \(\overline{f}\): \(K / Ker f \to Im f\) s predpisom \(\overline{f}(x + Ker f) := f(x)\)
        \end{center}
    \end{theorem}

    \begin{proof}
        \(Ker f \triangleleft K\) že vemo.
        \par
        Za \(u \in Ker f\) je \(f(x+u)=f(x)\), zato je definicija \(\overline{f}\) neodvisna od predstavnika razreda.
        \par
        \(\overline{f}\) je aditivna:
        \[\overline{f}((x + Ker f) + (y + Ker f)) = \overline{f}((x+y) + Ker f)\] \[= f(x+y) = f(x) + f(y) = \overline{f}(x + Ker f) + \overline{f}(y + Ker f)\]

        Za množenje opravimo analogni razmislek.

        \[Ker \overline{f} = \{x + Ker f\ |\ \overline{f}(x + Ker f) = 0 \} = \{x\ |\ f(x) = 0\} = \{0 + Ker f\}\]
        \(\overline{f}\) je torej injektivna.

        \[Im \overline{f} = Im f\]
        \(\overline{f}\) je surjektivna.
    \end{proof}

    \begin{proposition}
        Komutativen kolobar \(K\) je obseg natanko tedaj, ko nima pravih idealov (tj. edina ideala sta \((0)\) in \(K\)).
    \end{proposition}

    \begin{proof}[Dokaz \(\implies\)]
        Recimo, da je \(K\) obseg in \(I \triangleleft K\) neničelni ideal v \(K\) (\(I \neq (0)\)). Potem obstaja \(0 \neq x \in I\). Ker je \(x\) obrnljiv v \(K\), potem za vsak \(a \in K\) velja \(a = (a \cdot x^{-1}) \cdot x \in I\), torej je \(I = K\).
    \end{proof}

    \begin{proof}[Dokaz \(\Longleftarrow\)]
        Recimo, da \(K\) nima pravih idealov. Naj bo \(0 \neq a \in K\). Potem je \((a) = K \cdot a = a \cdot K \triangleleft K\). Ker v \(K\) ni pravih idealov, je \((a) = K\). Ker  \(K\) vsebuje enoto in je celoten generiran z \(a\), je \(a \cdot b = b \cdot a = 1\) za nek \(b \in K\). Sledi, da je \(a\) obrnljiv, in ker smo za \(a\) izbrali poljuben element \(K\), so vsi elementi v \(K\) obrnljivi. Torej je \(K\) obseg.
    \end{proof}

    Poiščimo ideale v \(K / I\). Za pomoč definirajmo funkcijo
    \begin{center}
        \(q\): \(K \to K/I\) \\
        \(x \mapsto x+I\) \\
    \end{center}

    \begin{proposition}
        Naj bo \(K\) komutativen kolobar in \(I \triangleleft K\).
        \begin{enumerate}[label=\alph*)]
            \item Če je \(J \triangleleft K\), potem je \(q(J) \triangleleft K / I\).
            \item Če je \(J \triangleleft K / I\), potem je \(q^{-1}(J) \triangleleft K\) in \(I \triangleleft q^{-1}(J)\) 
        \end{enumerate}
    \end{proposition}

    \begin{proof}[Dokaz \emph{(a)}]
        Vzamemo \(a \in K\) in \(x \in J\). Potem je \(a+I \in K / I\) in \(x + I \in q(J)\).
        \[(a+I)(x+I) = (a \cdot x) + I \in q(J)\]
    \end{proof}

    \begin{proof}[Dokaz \emph{(b)}]
        Vzamemo \(a \in K\) in \(x \in q^{-1}(J)\). Potem je \(a+I \in K / I\) in \(x + I \in J\).
        \[q(a \cdot x) = (a \cdot x) + I = (a+I)(x+I) \in J \implies ax \in q^{-1}(J)\]
    \end{proof}

    Pokazali smo, da so ideali v \(K / I\) natanko ideali oblike \(J / I\). \(J\) je torej natanko ideal v \(K\), ki vsebuje \(I\), hkrati pa je \(I \triangleleft J\).

    \begin{definition}[Maksimalni ideal]
        \textbf{Maksimalni ideal} v kolobarju \(K\) je pravi ideal, ki ni vsebovan v nobenem drugem pravem idealu.
    \end{definition}

    \begin{theorem}
        Naj bo \(K\) kolobar in \(I \triangleleft K\). \(K / I\) je obseg natanko takrat, ko je \(I\) maksimalni ideal v \(K\).
    \end{theorem}

    \begin{proof}
        Vemo, da je \(K / I\) obseg natanko takrat, ko nima pravih idealov. Ker nima pravih idealov, ni idealov v \(K\), ki bi vsebovali \(I\), torej je \(I\) maksimalen ideal.
    \end{proof}
    

    \begin{theorem}
        Naj bo \(R\) obseg. \(I \triangleleft R[x]\) je maksimalen natanko tedaj, ko je \(I = (p(x))\) za nek nerazcepen polinom \(p(x)\).
    \end{theorem}

    \begin{proof}[Dokaz \(\implies\)]
      Vzamemo \(p(x) = q(x) \cdot r(x) \in R[x]\), kjer je \(q\) nekonstanten nerazcepen polinom in \(r\) neničeln polinom. Ker je \(st\ p \ge st\ q\), je \((p) \triangleleft (q) \triangleleft R[x]\). 
    \end{proof}

    \begin{proof}[Dokaz \(\Longleftarrow\)]
        Naj bo \(I = (p)\) za nek polinom \(p \in R[x]\). Če ni maksimalen, je \((p) \triangleleft J \triangleleft R[x]\). Vsi ideali v \(R[x]\) so glavni, zato je \(J = (q)\) za nek \(q \in R[x]\). Za \(q\) lahko predpostavimo, da nerazcepen. Torej je \(p(x) = q(x) \cdot r(x)\) in \(st\ q < st\ p\), torej \(r\) ni konstanten. Sledi, da je \(p\) razcepen polinom.
    \end{proof}


    \pagebreak
    \subsection{Obsegi}

    Obseg je komutativen kolobar, v katerem so vsi neničelni elementi obrnljivi. Stvari, ki nas zanimajo pri kolobarjih, npr. ulomki, ideali, kvocienti itd., so pri obsegih precej nezanimive. Namesto tega se bomo ukvarjali z bolj zanimivimi pojmi.

    \subsubsection{Razširitve obsegov}

    \begin{definition}[Razširitev obsega]
        Če je \(K\) podobseg obsega \(F\), pravimo, da je \(F\) \textbf{razširitev} obsega \(K\) in pišemo \(K \le F\).
    \end{definition}

    \begin{proposition}
        Če je \(F\) razširitev obsega \(K\), je \(F\) vektorski prostor nad \(K\).
    \end{proposition}

    Dimenzijo \(K\)-vektorskega prostora \(F\) \(dim_{K}F\) običajno označimo z \([F:K]\). Če je dimenzija končna, pravimo, da je \(F\) \textbf{končne razširitev}, sicer pa je \textbf{neskončna razširitev} \(K\).

    \begin{theorem}
        Za obsege \(K \le F \le E\) velja \([E:K] = [E:F] \cdot [F:K]\).
    \end{theorem}

    \begin{proof}
        Za neskončne razširitve je očitno, zato se omejimo na končne razširitve.
        \par
        Naj bo \(x_1,...,x_m\) baza za \(F\) nad \(K\) in naj bo \(y_1,...,y_n\) baza za \(E\) nad \(F\). Pokazati moramo, da je \(\{x_i y_j\ |\ i = 1,...,m, j = 1,...,n\}\) baza \(E\) nad \(K\).
        \par
        Vzemimo \(e \in E\). \(e = f_1 y_1 + ... + f_n y_n\) za \(f_1,...,f_n \in F\). Obenem je \(f_i = k_{i1} x_1 + ... + k_{im} x_m\) za \(k_{ij} \in K\). Torej je \(e = \sum_{i = 1}^n \sum_{j = 1}^m k_{ij} x_j y_i\).
        \par
        Pokažimo tudi linearno neodvisnost baze. Recimo, da je linearno odvisna. Potem je \(0 = \sum_{i = 1}^n \sum_{j = 1}^m k_{ij} x_j y_i\). Ker so \(y_i\) linearno neodvisni, ke \(\sum_{j = 1}^m = 0\). Ker so tudi \(x_j\) linearno neodvisni, je \(k_{ij} = 0\) za vse \(i\) in \(j\). 
    \end{proof}

    Poglejmo si, kako izgleda najmanjša razširitev obsega \(K\), ki vsebuje nek \(a \in F\).
    
    \begin{proposition}
        \(\)\par
        \begin{enumerate}[label=\alph*)]
            \item Najmanjši podkolobar \(F\), ki vsebuje \(K\) in \(a \in F\), je \(K[a] = \{p(a)\ |\ p \in K[x]\}\).
            \item Najmanjši podobseg \(F\), ki vsebuje \(K\) in \(a \in F\), je \(K(a) = \{\frac{p(a)}{q(a)}\ |\ p,q \in K[x], q(a) \neq 0\}\).
        \end{enumerate}
    \end{proposition}

    \begin{proof}[Dokaz \emph{(a)}]
        Vsak kolobar, ki vsebuje \(K\) in \(a\), mora vsebovati tudi vse potence \(a\) in njihove \(K\)-linearne kombinacije \(k_0 + k_1 a + ... + k_n a^n\). Nadaljevanje dokaza izpuščeno.
    \end{proof}

    \begin{proof}[Dokaz \emph{(b)}]
        Obseg mora poleg \(K\)-linearnih kombinacij potenc \(a\) vsebovati še vse kvociente, katere predstavimo z ulomki oblike \(\frac{p(a)}{q(a)}\). Konstruiramo obseg ulomkov.
    \end{proof}

    Kolobar lahko razširimo z več elementi \(a_1,a_2,..\) naenkrat. Takšne razširitve označimo z \(K[a_1,a_2,...]\) za kolobarje in \(K(a_1,a_2,...)\) za obsege. Vrstni red dodanih elementov je lahko v zapisu poljuben.

    \begin{definition}[Enostavna razširitev]
        Razširitev obsega \(K\) je \textbf{enostavna}, če smo obsegu \(K\) dodali eden element \(a \in F\).
    \end{definition}

    Definiramo poseben homomorfizem kolobarjev:
    \begin{center}
        \(\phi_a\): \(K[x] \to F\) \\
        \(\phi_a\): \(p(x) \mapsto p(a)\)
    \end{center}

    Zanima nas, ali je \(\phi_a\) injektiven.
    \par
    Če je \(\phi_a\) injektiven, je \(Ker f = \{0\}\), \(a\) ni ničla nobenega (netrivialnega) polinoma s koeficienti v \(K\). Tedaj pravimo, da je \(a\) \textbf{transcendenten} nad \(K\).
    \par
    Če pa \(\phi_a\) ni injektiven, potem obstaja netrivialen polinom s koeficienti v \(K\), v katerem je \(a\) ničla. Tedaj pravimo, da je \(a\) \textbf{algebraičen} nad \(K\).
    
    \begin{definition}[Algebraična razširitev]
        Naj bo \(F\) razširitev nad obsegom \(K\). Če so vsi elementi \(F\) algebraični nad \(K\), pravimo, da je \(F\) \textbf{algebraična razširitev} nad \(K\).
    \end{definition}

    \begin{definition}[Transcendentna razširitev]
        Naj bo \(F\) razširitev nad obsegom \(K\). Če je vsaj eden element \(F\) transcendenten nad \(K\), je \(F\) \textbf{transcendentna razširitev} nad \(K\). 
    \end{definition}

    \begin{remark}
        Za dano število je običajno zelo težko dokazati, da je transcendentna nad \(\mathbb{Q}\). Na primer, transcendentnost \(e\) je bila dokazana leta 1873, transcendentnost \(\pi\) je bila dokazana leta 1882, še vedno pa ne vemo, ali je \(\pi + e\) transcendentna nad \(\mathbb{Q}\).
    \end{remark}

    \begin{theorem}
        Če je \(a\) transcendenten nad \(K \le F\), potem je \(K[a] \cong K[x]\) in \(K(a) \cong K(x)\).
    \end{theorem}

    Bolj zanimive so algebraične razširitve.
    \par
    Če je \(a \in F\) algebraičen nad \(K\), potem je \(Ker\ \phi_a\) pravi ideal v \(K[x]\). V \(K[x]\) so vso ideali glavni, zato je \(Ker\ \phi_a = (g)\) za nek polinom \(g \in K[x]\). Če dodatno zahtevamo, da je \(g\) moničen (vodilni koeficient je 1), potem je \(g\) enolično določen in mu pravimo \textbf{minimalni polinom} za element \(a\) nad \(K\) in ga označimo z \(g_a(x)\).

    \par
    Po izreku o izomorfizmu velja
    \[K[a] \cong K[x] / (g_a)\]

    \begin{example}
        \(\mathbb{Q}[\sqrt{2}] \cong \mathbb{Q}[x] / (x^2-2)\), ker je \(g_{\sqrt{2}}(x) = x^2-2\).
    \end{example}

    \begin{lemma}
        Ideal \((g_a) \triangleleft K[x]\) je maksimalen.
    \end{lemma}

    \begin{proof}
        Pokazati moramo, da je \(g_a\) nerazcepen. Če bi veljalo \(g_a(x) = p(x) q(x)\) za polinoma strogo nižje stopnje v \(K[x]\), bi iz \(0 = g_a(a) = p(a) q(a)\) dobili \(p \in Ker\ \phi_a\) ali \(q \in Ker\ \phi_a\). Ampak \(Ker\ phi_a = (g_a)\), zato je \(g_a\) polinom najmanjše stopnje v \(Ker\ \phi_a\). Ker sta \(p\) in \(q\) strogo manjše stopnje kot \(g_a\), ne moreta biti elementa \(Ker\ \phi_a\). Prišli smo do protislovja, torej je \(g_a\) nerazcepen. 
    \end{proof}

    \begin{corollary}
        \(K[x] / (g_a)\) je obseg, torej \(K(a) \cong K[a]\).
    \end{corollary}

    Zato je \(K[x] / (g_a)\) vektorski prostor z bazo \(1 + (g_a), x + (g_a),..., x^{n-1} + (g_a)\), kjer je \(n\) stopnja polinoma \(g_a\). Po izreku o izomorfizmu imamo izomorfizem \(\overline{\phi_a}\). Ta preslika bazo v elemente \(1,a,...,a^{n-1} \in F\).

    \par
    Ugotovitve povzemimo v naslednjem izreku.

    \begin{theorem}
        Naj bo \(a \in F\) algebraični element nad \(K \le F\)
        \begin{enumerate}[label=\alph*)]
            \item Obstaja natanko določen monični polinom \(g_a \in K[x]\), ki deli vse polinome, ki imajo \(a\) za ničlo.
            \item \(K(a) \cong K[a] \cong K[x] / (g_a)\)
            \item \([K(a) : K] = deg\ g_a\) je stopnja \(a\) nad \(K\) (pišemo \(deg_{K}\ a\)). Za bazo \(K(a)\) lahko vzamemo \(1,a,...,a^{n-1}\), kjer je \(n = deg_{K}\ a\).
        \end{enumerate}    
    \end{theorem}

    \begin{corollary}
        Če je \(F\) končna razširitev \(K\), potem za vsak \(a \in F\) velja \(deg_{K}\ a\ |\ [F:K]\)
    \end{corollary}

    \begin{proof}
        Iz \(K \le K(a) \le F\) sledi \([F:K] = [F:K(a)] \cdot [K(a):K] = [F:K(a)] \cdot deg_{K}\ a\).
    \end{proof}


    Videli smo, da so vse transcendentne razširitve neskončne, enostavne algebraične razširitve pa končne. Splošne algebraične razširitve so pa lahko tudi neskončne.

    \begin{theorem}
        \(\)\par
        \begin{enumerate}[label=\alph*)]
            \item Vsaka končna razširitev je algebraična.
            \item Naj bo \(K \le F\) razširitev obsega. Če je \(A \subseteq F\) podmnožica števil, ki so algebraična nad \(K\), potem je \(K(A)\) algebraična raširitev \(K\).
            \item Če je \(F\) algebraična razširitev \(K\) in je \(E\) algebraična razširitev \(F\), potem je \(E\) algebraična razširitev \(K\).
        \end{enumerate}
    \end{theorem}

    \begin{proof}[Dokaz \emph{(a)}]
        Naj bo \([F:K]\) in \(a \in F\). Potem je množica \(\{1,a,...,a^{n}\}\) linearno odvisna, torej obstaja netrivialna \(K\)-linearna kombinacija 
        \[k_0 1 + k_1 a + ... + k_n a^{n} = 0\]
        Torej je \(a\) algebraičen nad \(K\).
    \end{proof}

    \begin{proof}[Dokaz \emph{(b)}]
        Vsak element \(a \in K(A)\) se da zapisati kot 
        \[a = \frac{p(a_1,a_2,...,a_n)}{q(a_1,a_2,...,a_n)}\]
        za primerno izbrane polinome \(p,q \in K[x_1,x_2,...,x_n]\) in \(a_1,a_2,...,a_n \in A\). To pomeni, da je \(a \in K(a_1,...,a_n)\), ki je končna razširitev \(K\). Po (1) je \(a\) algebraična nad \(K\).
    \end{proof}

    \begin{proof}[Dokaz \emph{(c)}]
        Naj bo \(a \in E\). Po privzetku obstaja \(p(x)=a_0 + a_1 x + ... + a_n x^n \in F[x]\), za katerega je \(p(a) = 0\). To pomeni, da je \(a\) algebraičen nad \(K(a_1,...,a_n)\), ki je končna algebraična razširitev \(K\). Sledi, da je tudi \(a\) algebraičen nad \(K\).
    \end{proof}

    \begin{definition}[Algebraično zaprt obseg]
        Če dani obseg nima nobene prave algebraične razširitve oz. nad njim ni algebraičnih števil, je ta obseg \textbf{algebraično zaprt}.
    \end{definition}

    \begin{definition}[Algebraično zaprtje]
        Najmanjša razširitev obsega \(K\), ki je algebraično zaprta, je \textbf{algebraično zaprtje} obsega \(K\).
    \end{definition}
    
    \subsubsection{Razpadni obsegi}

    \begin{definition}[Razpadni obseg polinoma]
        Naj bo \(K\) obseg in \(p \in K[x]\). \textbf{Razpadni obseg polinoma} je najmanjša razširitev \(K\), ki vsebuje vse ničle polinoma \(p\).
    \end{definition}

    Privzemimo, da je \(p \in K[x]\) nerazcepen. Potem je \(K[x] / (p(x))\) razširitev, v kateri je \(a := x + (p(x))\) ničla polinoma \(p(x)\).
    \par
    \(p(x)\) lahko delimo z \((x-a)\) in dobimo kvocient s koeficienti v \(K[x] / (p(x))\). Postopek ponavljamo, dokler (po največ \(deg\ p\) korakih) ne dobimo razširitve \(F\), v kateri \(p(x)\) razpade na linearne faktorje. Lahko se zgodi, da smo dodali preveč ničel, zato vzamemo najmanjši podobseg \(F\), ki vsebuje \(K\) in vse ničle \(p(x)\).
    
    Končni rezultat je neodvisen od zaporedja razširitev, zato je razširitev enolično določena.
    \par
    Pokazali bomo, da vsak končni obseg dobimo kot razpadni obseg točno določenega polinoma nad \(\mathbb{Z}_p\).

    \subsubsection{Končni obsegi}

    Naj bo \(F\) končni obseg. Njegova karakteristika je \(p = char\ F\) in \(F\) je končno razsežni vektorski prostor nad \(\mathbb{Z}_p\). Sledi, da ima \(F\) natanko \(p^n\) elementov, kjer je \(n = [F:\mathbb{Z}_p]\).

    \begin{theorem}
        Za vsak \(n\) obstaja razširitev stopnje \(n\) obsega \(\mathbb{Z}_p\). Vsaka takšna razširitev je izomorfna razpadnemu obsegu polinoma \(x^{(p^n)} - x \in \mathbb{Z}_p[x]\).
    \end{theorem}

    \begin{proof}
        Opazimo, da ima \(x^{(p^n)} - x\) same različne ničle. Če bi imel večkratno ničlo, potem bi imel skupnega delitelja s svojim odvodom:
        \[(x^{(p^n)} - x)' = p^n x^{p^n - 1} - 1 \equiv -1\ (mod p)\]
        \(x^{p^n} - x\) ima \(n\) različnih ničel. Trdimo, da te ničle tvorijo obseg. Če \(x = x^{p^n}\) in \(y = y^{p^n}\), potem očitno enako velja tudi za \(x \cdot y\) in \(x / y\). Vendar tudi \(x \pm y\) ustrezata temu pogoju, ker je \((x \pm y)^{p^n} \equiv x^{p^n} \pm y^{p^n} (mod\ p)\).
        \par
        Sklepamo, da množica ničel obseg in sicer ravno razpadni obseg polinoma \(x^{p^n} - x\).
        \par
        Obratno, če ima \(F\) \(p^n\) elementov, potem elementi \(F\) zadoščajo enačbi \(x \cdot (x^{p^n} - 1) = 0\), torej \(F\) vsebuje vse ničle \(x^{p^n} - x\) in je po izreku o enoličnosti izomorfen \(\mathbb{Z}_p (x^{p^n} - x)\).
    \end{proof}

    \begin{definition}[Galoisov obseg]
        Končni obsegi moči \(p^n\) so \textbf{Galoisov obseg} \(GF(p^n)\).
    \end{definition}




    \pagebreak
    \section{Topologija: zveznost, kompaktnost in povezanost}

    \begin{definition}[Topološka struktura]
        Naj bo \(X\) poljubna množica. \textbf{Topološka struktura} ali krajše \textbf{topologija} na \(X\) je podana z množičo odprtih okolic, tj. družino \(\tau\) podmnožic \(X\), ki zadoščajo pogojema:
        \begin{enumerate}[label=(T\arabic*)]
            \item Poljubna unija množic iz \(\tau\) je v \(\tau\).
            \item Končen presek množic iz \(\tau\) je v \(\tau\). 
        \end{enumerate}

        Na kratko: \(\tau\) je zaprta za unije in končne preseke.
    \end{definition}

    Zakaj omejitev na končne preseke? Ob neskočnem preseku se lahko zgodi, da je zaprt.
    \[ \bigcup_{n=1}^{\infty}\ (-\frac{1}{n},\ \frac{1}{n}) = [0,1] \]

    Zahtevo \((T2)\) lahko poenostavimo: \(U,V \in \tau \implies U \cap V \in \tau\).
    
    \par
    Unija prazne družine je prazna, presek prazne družine pa cel \(X\), zato pogosto navedemo zahtevo \((T3)\ \emptyset, X \in \tau\).

    \begin{example}
        Pri analizi smo za podmnožico \(U \subseteq \mathbb{R}\) (oz. \(\mathbb{R}^n\)) rekli, da je odprta, če so vse njene točke notranje. Točka \(u \in U\) je notranja, če \(U\) vsebuje interval (ali kroglo) okoli točke \(u\). Ni se težko prepričati, da družina odprtih podmnožic \(\mathbb{R}\) (oz. \(\mathbb{R}^n\)) podaja topološko strukturo.
    \end{example}

    \begin{example}
        Za množico \(X\) definiramo funkcijo razdalje (metriko) 
        \begin{center}
            \(d\): \(X \times X \to [0, \infty)\)
        \end{center} 
        \((X,d)\) je metrični prostor. V metričnih prostorih definiramo krogle \(K(x_0,r)=\{\ x \in X\ |\ d(x_0,x) < r\ \}\), notranje točke in odprte množice opredelimo kot pri \(\mathbb{R}^n\). Tedaj je družina odprtih podmnožic topologija na \(X\). Pravimo ji \textbf{topologija porojena z metriko \(d\)} in jo včasih označimo z \(\tau_d\).
    \end{example}

    \begin{example}
        \(\tau=\{\ \emptyset, X\ \}\) je topologija na \(X\). Pravimo ji \textbf{trivialna topologija}. Nasprotna skrajnost je topologija, v katerem so vse množice (posebej singletoni) odprte. Pravimo ji \textbf{diskretna topologija}. Npr. običajna razdalja med točkami \(\mathbb{N}\) porodi diskretno topologijo.
    \end{example}

    \begin{definition}[Metrizabilna topologija]
        Topologija, porojene z metriko so \textbf{metrizabilne}. 
    \end{definition}
    Različne metrike pogosto porodijo isto topologijo, tj. isti pojem bližine.

    \par
    Komplementi odprtih množic so zaprte množice. Družina zaprtih množic 
    \[\mathscr{Z}=\{\ U^\complement\ |\ U \in \tau \}\]
    je zaprta za poljubne preseke in končne unije. Velja tudi \(\emptyset, X \in \mathscr{Z}\).

    \begin{definition}[Notranjost, zaprtje in rob množice]
        \(\)\par
        \begin{enumerate}[label=\arabic*)]
            \item Notranjost \(A\): \(Int A = \mathring{A} = \bigcup\ \{\ U \in \tau\ |\ U \subseteq A\ \}\) je največja odprta množica, vsebovana v \(A\).
            \item Zaprtje \(A\): \(Cl A = \overline{A} = \bigcap\ \{\ F \in \mathscr{Z}\ |\ F \supseteq A \}\) je najmanjša zaprta množica, ki vsebuje \(A\).
            \item Rob \(A\): \(Fr A = \dot{A} = Cl A - Int A\).
        \end{enumerate}
    \end{definition}


    \subsection{Zveznost}

    Po običajni intuiciji je funkcije zvezna, če slika točke, ki so si blizu, v točke, ki so si blizu.

    \begin{definition}
        Funkcija \(f\): \((X,\tau_X) \to (Y,\tau_Y)\) je zvezna, če za vsak \(V \in \tau_Y\) velja \(f^{-1}(V) \in \tau_X\).
    \end{definition}

    V metričnih prostorih se definicija ujema z običajno definicijo zveznosti, vendar je veliko preprostejša.

    \begin{theorem}
        Kompozitum zveznih funkcij je zvezna funkcija.
    \end{theorem}

    \begin{proof}
        Naj bosta \(f\): \((X,\tau_X) \to (Y,\tau_Y)\) in \(g\): \((Y,\tau_Y) \to (Z,\tau_Z)\) zvezni funkciji. Če je \(W \in \tau_Z\), je \(g^{-1}(W) \in \tau_Y\). Potem je tudi \(f^{-1}(g^{-1}(W)) = (gf)^{-1}(W) \in \tau_X\). Torej je \(g \circ f\) zvezna.
    \end{proof}

    Naslednji izrek je uporabna karakterizacija zveznosti:
    \begin{theorem}
        Naslednje trditve so ekvivalentne:
        \begin{enumerate}[label=\arabic*)]
            \item \(f\): \(X \to Y\) je zvezna.
            \item Za vsako odprto \(V \subseteq Y\) je \(f^{-1}(V)\) odprta v \(X\).
            \item Za vsako zaprto \(A \subseteq Y\) je \(f^{-1}(A)\) zaprto v \(X\).
            \item Za vsak \(A \subseteq X\) je \(f(\overline{A}) \subseteq \overline{f(A)}\).
        \end{enumerate}
    \end{theorem}

    \begin{proof}[Dokaz \emph{(1)} \(\Longleftrightarrow\) \emph{(2)}]
        je definicija zveznosti.
    \end{proof}

    \begin{proof}[Dokaz \emph{(2)} \(\Longleftrightarrow\) \emph{(2)}]
        \(f^{-1}(A^\complement) = (f^{-1}(A))^\complement\).    
    \end{proof}

    Zaprtje \(\overline{A}\) si predstavljamo kot točke \(X\), ki so tako blizu, da jih topologija ne loči od \(A\).
    \par
    \((4)\) preberemo kot "zveznost pomeni, da se točke, ki so blizu A, preslikajo v točke, ki so blizu \(f(A)\)."

    \par
    Pri dokazu \((3)\) \(\Longleftrightarrow\) \((4)\) bomo uporabili zvezi \(f^{-1}(f (A)) \supseteq A\) in \(f(f^{-1} (B)) = B\).

    \begin{proof}[Dokaz \emph{(3)} \(\implies\) \emph{(4)}]
        Naj bo \(A \subseteq X\). \(A \subseteq f^{-1}(f (A)) \subseteq f^{-1}( \overline{f(A)}) \). \(\overline{f(A)}\) je zaprtje \(f(A)\), torej je zaprta. Zaradi \((3)\) je tudi \(f^{-1}(\overline{f(A)})\) zaprta. Sledi \(\overline{A} \subseteq f^{-1} (\overline{f(A)}) \). Iz tega sledi \(f(\overline{A}) \subseteq \overline{f(A)}\). 
        
    \end{proof}

    \begin{proof}[Dokaz \emph{(4)} \(\implies\) \emph{(3)}]
        Vzemimo zaprto množico \(B \subseteq Y\). Velja \(B = \overline{B} = \overline{f(f^{-1}(B))}\). Po \((3)\) je \(\overline{f(f^{-1}(B))} \subseteq f(\overline{f^{-1}(B)})\). Iz tega sledi \(\overline{f^{-1}(B)} \subseteq f^{-1}(B)\). To je mogoče je, če je \(\overline{f^{-1}(B)} = f^{-1}(B)\), torej je \(f^{-1}(B)\) zaprta.
    \end{proof}

    \subsection{Homeomorfizmi}

    Zvezne funkcije so analogne homomorfizmom med grupami ali kolobarji. Izomorfizmi so bijektivni homomorfizmi, ki ponavadi kažejo kongruenco oz. enakost dveh struktur. V topologiji pa bijektivnost ni dovolj - za zvezno bijektivno funkcijo \(f\) ni nujno, da je \(f^{-1}\) tudi zvezna. Zato zahtevamo, da je \(f^{-1}\) zvezna in v izogib dvoumnosti uvedemo nov pojem.
    
    \begin{definition}[Homeomorfizem]
        Naj bo \(f\): \(X \to Y\) bijektivna zvezna funkcija med topologijama. Če je \(f^{-1}\) tudi zvezna, je \(f\) \textbf{Homeomorfizem}.

        \par
        Če med topologijama \((X,\tau_X)\) in \((Y,\tau_Y)\) obstaja homeomorfizem, pravimo, da sta \textbf{homeomorfna} in zapišemo \((X,\tau_X) \approx (Y,\tau_Y)\) oz. \(X \approx Y\).
    \end{definition}

    \begin{example}
        \([a,b] \approx [c,d]\):
        \[y = \frac{d - c}{b - a} (x-a) + c\]
        Funkcija je linearna, torej zvezna. Inverz je tudi linearen, zato tudi zvezen. Podobno je tudi \((a,b) \approx (c,d)\) in \([a,b) \approx [c,d)\).
    \end{example}

    \begin{example}
        \((0,1) \approx \mathbb{R} \):
        \begin{center}
            \(\tan\): \((-\frac{\pi}{2},\frac{\pi}{2}) \to \mathbb{R}\) zvezna, bijektivna
        \end{center}
        Inverz je \(\arctan\), ki je tudi zvezen. Komponiramo z \((0,1) \approx (-\frac{\pi}{2},\frac{\pi}{2})\).
        \par
        Pogosto uporabimo tudi sledeči homeomorfizem \((-1,1) \approx \mathbb{R}\):
        \begin{center}
            \(f(x) = \frac{x}{\sqrt{1-x^2}}\) in njegov inverz \(g(x) = \frac{x}{\sqrt{1+x^2}}\)
        \end{center}
        Prednost tega homeomorfizma je, da ga z lahkoto razširimo na več dimenzij:
        \begin{center}
            \(B^n =\) enotska krogla v \(\mathbb{R}^n\) \\
            \(f(\vec{x}) = \frac{\vec{x}}{\sqrt{1 - \parallel x \parallel^2}}\) \\
            \(g(\vec{x}) = \frac{\vec{x}}{\sqrt{1 + \parallel x \parallel^2}}\)
        \end{center}
    \end{example}

    \begin{example}
        Rob enotske krogle \(B^n\) je enotska sfera 
        \[S^{n-1} = \{\ x \in \mathbb{R}^n\ |\ \parallel x \parallel\ = 1\ \}\]
        Poiščimo homeomorfizem \(S^1 - \{\ (0,1)\ \} \approx \mathbb{R}\).

        \par
        Ideja: krožnico brez vrhnje točke projeciramo na abscisno os.
        \[(x,y) \mapsto (u,0)\]
        Kako dobimo \(u\)? 
        \par
        Vzamemo \(x,y \in S^1\): \((0,1), (x,y),(u,0)\) kolinearne, zato je \(\frac{x-0}{y-1} = \frac{u}{-1}\), torej \(u = \frac{x}{1-y}\).
        \begin{center}
            \(f(x,y) = \frac{x}{1 - y}\) zvezna za \(y \neq 1\)
        \end{center}
        Inverz: premica skozi \((0,1)\) in \((u,0)\) seka krožnico v \(x,y\):
        \[\frac{x}{u} + y = 1\]
        \[x^2 + y^2 = 1\]

        Iz tega dobimo:
        \[y = 1 - \frac{x}{u}\]
        \[x^2 + (1 - \frac{x}{u})^2 = 1\]
        \[x^2 (1 + \frac{1}{u^2}) = \frac{2x}{u}\]

        Iz zadnje enačbe sledi:
        \[x = \frac{2}{u(1+\frac{1}{u^2})} = \frac{2u}{u^2 + 1}\]
        \[y = \frac{u^2 - 1}{u^2 + 1}\]

        Inverz \(f\) je torej:
        \[g(u) = (\frac{2u}{u^2 + 1}, \frac{u^2 - 1}{u^2 + 1})\]

        Tudi tu imamo posplošitev na višje dimenzije:
        \[S^n - \{\ (0,0,...,1)\ \} \xleftrightharpoons[g]{f} \mathbb{R}\]
        \[f(x_1,x_2,...,x_{n+1}) := \frac{1}{1-x_{n-1}} (x_1,...,x_n)\]
        \[f(\vec{x}, y) = \frac{\vec{x}}{1-y}\]
        \[g(\vec{x}) = (\frac{2\vec{x}}{\parallel \vec{x} \parallel^2 + 1}, \frac{\parallel \vec{x} \parallel^2 - 1}{\parallel \vec{x} \parallel^2 + 1})\]

        Preslikavi \(f\) pravimo \textbf{stereografska projekcija}. Posebno pomembna je za \(n=2\): tedaj dobimo \(S^2 \approx \mathbb{C} \cup \{ \infty \}\)
    \end{example}

    Računanje inverzov je velikokrat nepraktično. Na srečo lahko pokažemo, da je zvezna \(f\) homeomorfizem, brez da bi se sklicevali na konkreten \(f^{-1}\).

    \begin{definition}
        Zvezna funkcija \(f\) je \textbf{odprta}, če je slika vsake odprte množice odprta.
        \par
        Zvezna funkcija \(f\) je \textbf{zaprta}, če je slika vsake zaprte množice zaprta.
    \end{definition}

    \(f^{-1}\ je\ zvezna\ \Longleftrightarrow\) \((U^{odprta} \subseteq X \implies (f^{-1})^{-1} (U) = f(U)\ odprta\ v\ Y)\).
    \par
    \(f^{-1}\ je\ zvezna\ \Longleftrightarrow\) \((A^{zaprta} \subseteq X \implies f(A)\ zaprta\ v\ Y)\).

    
    \begin{theorem}
        Naslednje trditve so ekvivalentne:
        \begin{enumerate}[label=(\arabic*)]
            \item \(f\): \(X \to Y\) je homeomorfizem.
            \item \(f\): \(X \to Y\) je zvezna bijekcija in \(f^{-1}\) je zvezna.
            \item \(f\): \(X \to Y\) je zvezna, odprta bijekcija.
            \item \(f\): \(X \to Y\) je zvezna, zaprta bijekcija.
        \end{enumerate}
    \end{theorem}

    \subsection{Kompaktnost}

    \begin{definition}
        Naj bo \(X,\tau_X\) topološki prostor. \textbf{Odprto pokritje} množice \(X\) je družina \(\mathscr{U} \in \tau\), katere unija je celoten \(X\).
    \end{definition}

    \begin{example}
        Pokritje \(\mathbb{R}\) z odprtimi intervali.
    \end{example}

    \begin{example}
        Pokritje \(\mathbb{R}^2\) s pravokotniki \((a,b) \times *(c,d)\)
    \end{example}


    \begin{definition}
        Prostor \(X\) je \textbf{kompaktnem}, če v vsakem odprtem pokritju obstaja končo podpokritje (tj. končna poddružina, ki tudi pokrije \(X\)).
    \end{definition}

    Malo splošneje, \(A \subseteq X\) je kompakten, če za vsako odprto pokritje \(A\) končno podpokritje.

    \begin{example}
        Vsaka končna množica je kompaktna.
    \end{example}

    \begin{proposition}
        V metričnem prostoru je vsaka kompaktna množica omejena.
    \end{proposition}

    \begin{proof}
        Naj bo \(X\) metrični prostor.
        \[X \subseteq K(x_0,1) \cup K(x_0,2) \cup ... \]
        Kompaktnost pomeni, da je \(X\) pokrit z nekim končnim naborom krogel. Ker so vse krogle omejene, je tudi \(X\) omejen. 
    \end{proof}

    \begin{example}
        \((0,1)\) \underline{ni} kompakten.
        \par
        To je zato, ker \(\{\ (\frac{1}{n},1)\ |\ n=2,3,...\ \}\) nima končnega podpokritja.
        \par
        Alternativno: kompaktnost je topološka lastnost in \((0,1) \approx \mathbb{R}\), ki je neomejen, torej nekompakten.
        \par
        Podobno tudi \([0,1) \approx [0,\infty)\) ni kompakten.
    \end{example}

    \begin{theorem}
        Interval \([a,b]\) je kompakten.
    \end{theorem}

    \begin{proof}
        Naj bo \(\mathscr{U}\) odprto pokritje za \([a,b]\).
        \par
        \(\ \{\ x\ |\ [a,x]\) je pokrit s končno mnogo množicami v \(\mathscr{U}\ \} \supseteq [a,b]\) ima supremum \(c\).
        \par
        Recimo, da je \(c < b\). \(c \in U \in \mathscr{U}\). Obstaja takšen \(c' \le c\), da je \([a,c']\) pokrit z množicami \(U_1,...U_n\). Ampak potem je tudi \((U_1 \cup ... \cup U_n) \cup U\) pokrit s končno množicami v \(\mathscr{U}\), zato \(c\) ne more biti supremum. Torej je \(c = b\).
    \end{proof}

    \begin{theorem}
        V metričnem prostoru je vsaka kompaktna množica zaprta.
    \end{theorem}

    \begin{proof}
        Naj bo \(K\) kompaktna podmnožica metričnega prostora in \(x_0 \notin K\). Pokazati moramo, da je \(x_0\) zunanja točka za \(K\), tj. ima okolico, ki ne seka \(K\), oziroma za vsak \(x \in K\) obstaja dovolj majhen \(r_x \ge 0\), da se \(K(x,r_x)\) in \(K(x_o,r_x)\) ne sekata.
        \par
        Ekvivalentno, moramo pokazati, da je \(\displaystyle K \subseteq \bigcup_{x \in K} K(x,r_x)\), ki ne seka \(\displaystyle \bigcap_{x \in K} K(x_o,r_x)\). Vendar \(\displaystyle \bigcap_{x \in K} K(x_o,r_x)\) ni nujno odprta. Ker je \(K\) kompakten, lahko najdemo takšne \(x_1,...,x_n \in K\), da \(K \subseteq K(x_1,r_{x_1}) \cup ... \cup K(x_n,r_{x_n})\), ki ne seka \(K(x_o,r_{x_1}) \cap ... \cap K(x_0,r_{x_n})\), saj smo izbrali takšne \(r_{x_i}\), da se \(K(x_i,r_{x_i})\) in \(K(x_o,r_{x_i})\) ne sekata. To je odprta okolica \(x_0\), ki ne seka \(K\).
    \end{proof}

    \begin{theorem}
        Naj bo \(X\) kompaktna množica in \(A \subseteq X\) zaprta. Potem je tudi \(A\) kompaktna. 
    \end{theorem}

    \begin{proof}
        Naj bo \(\mathscr{U}\) odprto pokritje \(A\). \(\mathscr{U} \cup \{\ X - A\ \}\) je potem odprto pokritje \(X\).
        \par
        Ker je \(X\) kompaktna, lahko v \(\mathscr{U}\) najdemo končno podpokritje \(U_1,..,U_n,X-A\) za \(X\). Če iz tega pokritja vzamemo izpustimo \(X-A\), dobimo končno podpokritje za \(A\).
    \end{proof}

    Naslednji izrek je podan brez dokaza. Potrebovali ga bomo za izrek, ki sledi. 
    \begin{theorem}
        Če so \(X_1,...,X_n\) kompaktne, je \(X_1 \times ... \times X_n\) kompakten.
    \end{theorem}

    \begin{theorem}[Heine-Borelov izrek]
        \(A \subseteq \mathbb{R}^n\) je kompaktna natanko takrat, ko je zaprta in omejena.
    \end{theorem}

    \begin{proof}[Dokaz \(\implies\)]
        Smo že.
    \end{proof}

    \begin{proof}[Dokaz \(\Longleftarrow\)]
        Ker je \(A\) omejena, je \(A \subseteq [a_1,b_1] \times ... \times [a_n, b_n]\) za dovolj velik kompakten kvader (kompakten je, ker je kartezični produkt zaprtih intervalov, za katere vemo, da so kompaktni).
        \par
        Ker je \(A\) tudi zaprta in je podmnožica kompaktne množice, je \(A\) kompaktna. 
    \end{proof}


    \begin{theorem}
        V kompaktnem prostoru ima vsaka neskončna množica stekališče (tj. obstaja takšna točka \(a\), da ima vsaka njena okolica neskončno točk).
    \end{theorem}

    \begin{proof}
        Naj bo \(A \in K\) neskončna množica v kompaktnem prostoru. Denimo, da \(A\) nima stekališča. Potem ima vsak \(x \in X\) okolico \(U_x\), ki vsebuje le končno mnogo elementov \(A\). 
        \par
        \(\{\ U_x\ |\ x \in X\ \}\) je odprto pokritje za \(X\). Ker je \(X\) kompakten, obstaja končno podpokritje \(U_1,...,U_n\), ki pokrije \(X\), torej je \(A \subseteq U_1 \cup ... \cup U_n\). Sledi, da je \(A\) kompaktna.
    \end{proof}

    \begin{corollary}[Bolzano-Weierstrass]
        Vsako omejeno zaporedje v \(\mathbb{R}^n\) ima konvergentno podzaporedje.
    \end{corollary}

    \begin{proof}
        Naj bo \(\{\ x_i\ \}\) omejeno zaporedje v nekem kompaktnem kvadru. Potem ima \(\{\ x_i\ \}\) stekališče \(s\). Če vzamemo točke zaporedje, ki gredo proti \(s\). To podzaporedje je očitno konvergentno.
    \end{proof}


    \begin{theorem}
        Naj bo \(f\): \(X \to Y\) zvezna funkcija in \(A \subseteq X\) kompaktna podmnožica. Potem je \(f(A) \subseteq Y\) kompaktna.
    \end{theorem}

    \begin{proof}
        Naj bo \(\mathscr{U}\) odprto pokritje \(f(A)\). Potem je \(\{\ f^{-1}(U)\ |\ U \in \mathscr{U}\ \}\) odprto pokritje za \(A\). Ker je \(A\) kompaktna, obstaja končno podpokritje \(f^{-1}(U_1),...,f^{-1}(U_n)\) za \(A\). Torej je \(U_1,...,U_n\) končno pokritje za \(f(A)\). \(f(A)\) je torej kompaktna.
    \end{proof}

    \begin{corollary}
        Naj bo \(X\) kompakten prostor. Potem vsaka zvezna \(f\): \(X \to \mathbb{R}\) zavzame svoj minimum in maksimum.
    \end{corollary}

    \begin{proof}
        \(f(X)\) je kompaktna, torej je omejena in zaprta. Torej obstaja \(inf \{\ f(x)\ |\ x \in X\ \}\), ki je v \(f(X)\), torej \(f\) zavzema minimum. Analogno za maksimum.
    \end{proof}

    Opis kompaktnosti z zaprtimi množicami:
    \begin{center}
        \(\mathscr{U}\) odprto pokritje, \(\displaystyle \bigcup_{U_\lambda \in \mathscr{U}} U_\lambda = X\) \(\longleftrightarrow\) \(\{\ U_\lambda^\complement\ |\ U_\lambda \in \mathscr{U} \}\) zaprte, \(\displaystyle \bigcap_{U_\lambda \in \mathscr{U}} U_\lambda^\complement = \emptyset\)
    \end{center}
    \begin{center}
        \(X = U_1 \cup ... \cup U_n \Longleftrightarrow U_1^\complement \cap ... \cap U_n^\complement = \emptyset\)
    \end{center}

    \begin{theorem}
        Prostor \(X\) je kompakten natanko tedaj, ko v vsaki družini zaprtih podmnožic, ki ima prazen presek, obstaja končna poddružina s praznim presekom.
    \end{theorem}

    \begin{corollary}[Cantorjev izrek o sendviču]
        \([a_1,b_1] \supseteq [a_2,b_2] \supseteq ...\), presek je neprazen.
    \end{corollary}


    \subsection{Povezanost}

    \begin{definition}[Separacija, povezana množica]
        \textbf{Separacija} množice \(X\) je razdelitev \(X = A \cup B\) na dve neprazni odprti disjunktni množici. Množica \(X\) je \textbf{povezana}, če nima separacije, sicer je \textbf{nepovezana}.
    \end{definition}

    Alternativni karakterizaciji:
    \begin{enumerate}[label=(\arabic*)]
        \item \(X\) ni mogoče zapreti kot unijo dveh zaprtih disjunktnih množic.
        \item V \(X\) ne obstaja \(A \neq \emptyset, X\), ki je hkrati odprta in zaprta.
    \end{enumerate}

    \begin{theorem}
        Povezane množice v \(\mathbb{R}\) so natanko intervali.
    \end{theorem}

    Opis intervala ne glede na to, ali vsebuje krajišča:
    \par
    \(I\) je interval, če iz \(a,b \in I\) in \(a < c < b\) sledi \(c \in I\).
    \begin{proof}[Dokaz \(\implies\)]
        Če \(I \subseteq \mathbb{R}\) ni interval, potem obstajajo \(a < c < b\), da velja \(a,b \in I\) in \(c \notin I\). Tedaj je \((- \infty, c) \cap I, (c, + \infty) \cap I)\) separacija \(I\), torej ni povezana. 
    \end{proof}

    \begin{proof}[Dokaz \(\Longleftarrow\)]
        Če interval \(I\) ni povezan, potem obstaja separacija \(I = A \cup B\), \(A,B\) odprti in neprazni. Vzamemo takšna \(a \in A\) in \(b \in B\), da je \(a < b\) (sicer zamenjamo \(A\) in \(B\)). Naj bo \(c := sup\{\ x\ |\ [a,x) \subseteq A\ \}\).
        \par
        Očitno je \(a \le c \le b\), torej je \(c \in I\). Poleg tega vsaka okolica \(c\) seka tako \(A\) kot \(B\), zato \(c\) ni notranja v nobeni, zato \(c \notin A\) in \(c \notin B\). Protislovje.
    \end{proof}

    
    \begin{theorem}
        Naj bo \(f\): \(X \to Y\) zvezna funkcija. Če je \(X\) povezana, je tudi \(f(X)\) povezana.
    \end{theorem}

    \begin{proof}
        Denimo, da je \(f(X) = A \cup B\). Potem je \(X = f^{-1}(A) \cup f^{-1}(B)\).
    \end{proof}

    \begin{corollary}[Izrek o vmesni vrednosti]
        Naj bo \(X\) povezana in \(f\): \(X \to Y\) zvezna. Če je \(a,b \in f(X)\), potem je \((a,b) \subseteq f(X)\).
    \end{corollary}


    \begin{definition}[Povezanost s potmi]
        Prostor \(X\) je \textbf{povezan s potmi}, če za poljubna \(x,y \in X\) obstaja pot \(p\): \([0,1] \to X\), \(p(0) = x\) in \(p(1) = y\).
    \end{definition}

    \begin{proposition}
        Če je \(X\) povezan s potmi, je \(X\) povezan.
    \end{proposition}

    \begin{proof}
        Recimo, da \(X\) ni povezan. Potem obstaja separacija \(X = A \cup B\). Vzemimo pot \(p\) od \(a \in A\) do \(b \in B\). Potem je \(f^{-1}(A)\), \(f^{-1}(B)\) separacija \([0,1]\). Protislovje.
    \end{proof}

    \begin{theorem}
        Če je \(M \subseteq \mathbb{R}^n\) povezana in odprta, potem je \(M\) povezana s potmi.
    \end{theorem}

    \begin{proof}
        Izberimo \(x_o \in M\) in definiramo:
        \begin{center}
            \begin{tabular}{c c}
                \(A=\) & \(\{\ x \in M\ | \) v \(M\) obstaja pot od \(x_0\) do \(x\) \(\}\) \\
               \(B=\) & \(\{\ x \in M\ | \) v \(M\) ni poti od \(x_0\) do \(x\) \(\}\) 
            \end{tabular}
            
        \end{center}

        Množica \(A\) je odprta: če obstaja pot \(x \in A\), jo lahko podaljšamo do vseh točk v krogli okoli \(x\).
        \par
        Množica \(B\) je tudi odprta: če bi lahko s potjo prišli do neke točke v okolici \(x\), bi lahko prišli tudi do točke \(x\).
        \par
        Torej je \(M = A \cup B\). \(A\) in \(B\) sta obe odprti, ampak \(M\) nima separacije, zato mora biti ena izmed \(A\) in \(B\) prazna. Očitno je \(x_0 \in A\), zato je \(B = \emptyset\).
    \end{proof}



    \section{Fourierova vrsta in transformacija}

    Na intervalu \([- \pi, \pi]\) želimo zvezno realno funkcijo zapisati kot vsoto sinusoid.
    
    \par
    Najprej pokažimo naslednji integral, ki nam bo v pomoč:
    \[\int_{- \pi}^{\pi} sin(mx) sin(nx) dx = 
        \begin{cases}
            0, & \quad{m \neq n} \\
            \pi, & \quad{m = n}
        \end{cases}
    \]

    Namreč: \(2 sin(mx) sin(nx) = cos((m-n)x) - cos((m+n)x)\), zato je:
    \[\int_{- \pi}^{\pi} sin(mx) sin(nx) dx = \frac{1}{2} (\frac{-sin((m-n)x}{m-n} \Big |_{- \pi}^{\pi} + \frac{sin((m+n)x)}{m+n} \Big |_{- \pi}^{\pi}) = 0\]
    Razen, ko je \(m = n\), ker je 
    \[\displaystyle \int_{- \pi}^{\pi} \frac{1}{2} cos(0)dx = \frac{1}{2} \int_{- \pi}^{\pi} 1 dx = \pi\]
    
    \par
    Naj bo \(f(x) = a_1 sin(x) + a_2 sin(2x) + ... = a_n sin(n)\).

    \[\displaystyle \int_{- \pi}^{\pi} f(x) sin(kx) dx = a_1 \int_{- \pi}^{\pi} sin(x) sin(kx) dx + ... + a_n \int_{- \pi}^{\pi} sin(nx) sin(kx) dx = a_k \pi\].

    Sledi, da je 
    \[a_k = \frac{1}{\pi} \int_{- \pi}^{\pi} f(x) sin(kx) dx\]

    To lahko naredimo tudi za poljubno integrabilno funkcijo.
    \[s(x) = a_1 sin(x) + a_2 sin(2x) + ... + a_n sin(nx)\]

    Ni nujno, da je \(s(x) = f(x)\) na \([- \pi, \pi]\), saj je \(s(x)\) soda, \(f(x)\) pa je poljubna. To lahko popravimo: vključimo še kosinuse, ki so lihe funkcije. Potrebujemo formuli:
    \[
        \int_{- \pi}^{\pi} cos(mx) cos(nx) dx = 
        \begin{cases}
            0, & \quad{m \neq n} \\
            \pi, & \quad{m = n \neq 0} \\
            2 \pi, & \quad{m = n = 0}
        \end{cases}
    \]
    \[
        \int_{- \pi}^{\pi} cos(mx) sin(nx) dx = 0
    \]
    
    S tem dobimo naslednje formule.
    \begin{theorem}
        Če je \(f(x) = \frac{a_o}{2} + a_1 cos(x) + ... + a_n cos(nx) + b_1 sin(x) + ... + b_n sin(nx) \), potem koeficiente \(a_k\) in \(b_k\) dobimo s formulama:
        \[a_k = \frac{1}{\pi} \int_{- \pi}^{\pi} f(x) cos(kx) dx\]
        \[b_k = \frac{1}{\pi} \int_{- \pi}^{\pi} f(x) sin(kx) dx\]
    \end{theorem}

    \begin{theorem}
        Če je \(f(x)\) poljubna funkcija, lahko uporabim zgornji formuli in dobimo vrsto:
        \[s(x) = \frac{a_0}{2} + (a_1 cos(x) + b_1 sin(x)) + (a_2 cos(x) + b_2 sin(x)) + ...\]
    \end{theorem}

    Funkcije \(cos(kx)\) in \(sin(kx)\) so absolutno omejene z 1, zato je za konvergenco dovolj, če konvergira vrsta 
    \[\frac{a_0}{2} + \sum_{k=1}^{\infty} (|a_k| + |b_k|)\]

    Ocena velikosti \(a_k\) z integracijo po delih (računanje izpuščeno):
    \[\pi a_k = \frac{1}{k^2} (f'(x) cos(kx) \Big|_{- \pi}^{\pi} - \int_{- \pi}^{\pi} f'' (x) cos(kx) dx)\]

    Privzemimo, da \(f\) izpolnjuje vse potrebne pogoje, tj. da je dvakrat zvezno odvedljiva in \(2 \pi\)-periodična (in je zato \(f' (- \pi) = f' (\pi) \)).
    \par
    Zaradi zveznosti drugega odvoda je \(|f'' (x) \le M|\) na \([- \pi, \pi]\), torej je:
    \[|a_k| \le \frac{2M}{k^2}\]

    Posledično vrsta
    \[\frac{a_0}{2} + \sum_{k=1}^{\infty} (a_k cos(kx) + b_k sin(kx))\]
    enakomerno konvergira proti neki zvezni \(2\pi\)-periodični funkciji.

    \par
    To lahko posplošimo na vse integrabilne funkcije:
    \begin{theorem}
        Naj bo \(f(x)\) funkcija z naslednjimi lastnostmi:
        \begin{itemize}
            \item \(2\pi\)-periodična
            \item odsekoma zvezna
            \item v vsaki točki ima levi in desni odvod.
        \end{itemize}
        Potem je njena Fourierova vrsta konvergentna. Vrednost te vrste je enaka \(f(x)\) v vseh točkah, kjer je \(f\) zvezna, v ostalih pa je enaka povprečju med levo in desno limito.
    \end{theorem}

    \begin{example}
        Fourierova vrsta od \(f(x) = x\).
        \par
        Funkcija je liha. Torej je \(a_n = 0\).
        \[x = 2 (sin(x) - \frac{sin(2x)}{2} + \frac{sin(3x){3}} - \frac{sin(4x)}{4} + ...)\]
    \end{example}

    Fourierovo vrsto lahko zapišemo bolj simetrično kot kompleksno vrsto:
    \[e^{ix} = cos(x) + isin(x)\]
    \[e^{-ix} = cos(x) - isin(x)\]
    
    \[\frac{a_0}{2} + \sum_{- \infty}^{\infty} (\frac{a_n - ib_n}{2} e^{inx} + \frac{a_n + ib_n}{2}e^{-inx}) = \sum_{- \infty}^{\infty} c_n e^{inx} \]
    
    \[c_0 = \frac{a_0}{2}\]
    \[c_n = \frac{a_n - ib_n}{2}\]
    \[c_{-n} = \frac{a_n + ib_n}{2}\]

    Kompleksne koeficiente lahko izračunamo direktno:
    \[c_n = \frac{1}{\pi} \int_{- \pi}^{\pi} f(x) e^{-inx} dx,\ n \in \mathbb{N}\]

    Želimo pridobiti amplitude in frekvence, ki sestavljajo \(2\pi\)-periodično funkcijo.
    \[\frac{1}{\pi} \int_{- \pi}^{\pi} f(x) e^{-inx} dx\]
    je amplituda \(e^{inx}\) v 
    \[f(x) = \sum_{- \infty}^{\infty} c_n e^{inx}\]

    Naslednja formula nam bo dala amplitude vseh frekvence:
    \[\int_{- \infty}^{\infty} f(t) e^{-iwt} dt = \hat{f} (w)\]
    \(\hat{f}(w)\) amplituda nihaja s frekvenco \(w\) v funkciji \(f(t)\).

    Lastnosti Fourierove transformacije:
    \begin{enumerate}
        \item Da pridemo do Fourierove transformacije funkcije, mora biti funkcija absolutno integrabilna.
        \item \(\lim_{w \pm \infty} f(w) = 0\), tj. amplitude pri zelo visokih frekvencah gredo proti 0.
        \item Linearnost: \(\hat{af+bg} = a\hat{f}+b\hat{g}\)
        
        \item Razteg: 
        \[\int_{- \infty}^{\infty} f(at) e^{-iwt} dt = \frac{1}{a} \int_{- \infty}^{\infty} f(s) e^{-i \frac{w}{a}s} ds = \frac{1}{a} \hat{f}(\frac{w}{a}) \]
        \[\hat{f(at)} = \frac{1}{|a|} \hat{f}(\frac{w}{a})\]

        \item Premik:
        
        \[ \hat{f(t-a)} = e^{-iwa} \hat{f}(w) \]

        \item Kompleksno konjugiranje:
        
        \[ \hat{\overline{f(t)}} = \overline{\hat{f}(-t)} \]

        \item Množenje s sinusom/kosinusom:
        
        \[ \hat{f(t) e^{iw_0 t}} = \hat{f}(w-w_0) \]

        \item Transformacija odvoda:
        
        \[ \hat{f' (t)} = iw \hat{f}(w) \]
        \[ \hat{f^{(n)}}(t) = (iw)^n \hat{f}(t) \]

        \item Odvod transformiranke:
        
        \[ \hat{(it)^n f(t)} = \hat{f}^{(n)} (w) \]

        \item Inverzna transformacija:
        
        \[ f(t) = \frac{1}{2\pi} \int_{- \infty}^{\infty} \hat{f}(w) e^{iwt} dw \]
        

    \end{enumerate}


    \subsubsection{Konvolucija}

    \[(a \ast b)(k) = \sum_{i} a(i) b(k - i)\]
    \[(f \ast g)(t) = \int_{- \infty}^{\infty} f(s) g(t - s) ds\]
    \[\hat{(f \ast g)}(w) = \hat{f}(w) \cdot \hat{g}(w)\]
\end{document}