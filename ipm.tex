\documentclass[11pt, a4paper]{article}
\usepackage{mathtools}
\usepackage{caption}
\usepackage{float}
\usepackage[slovene]{babel}
\usepackage{amssymb}
\usepackage{amsthm}
\usepackage{enumitem}
\usepackage{mathrsfs}



\hyphenpenalty=10000

\begin{document}
    \newtheorem{theorem}{Izrek}[section]
    \newtheorem{definition}[theorem]{Definicija}
    \newtheorem{corollary}[theorem]{Posledica}
    \newtheorem{lemma}[theorem]{Lema}
    \newtheorem{proposition}[theorem]{Trditev}
    \newtheorem{example}[theorem]{Zgled}

    \newtheorem*{remark}{Opomba}


    \title{Izbrana poglavja iz matematike}
    \author{Napisal Jure Pustoslemšek po zapiskih predavanj prof. dr. Petarja Pavešiča}
    \date{Junij 2020}
    \maketitle

    \section{Abstraktna algebra: kolobarji in obsegi}

    \begin{definition}[Kolobar]
        Kolobar je množica, kateri smo priredili notranji operaciji seštevanja in množenja, ki zadostujeta spodnjim kriterijem. Eksplicitno ga lahko zapišemo kot \((K,+,\cdot)\).
        \par
        Pri \underline{seštevanju} velja \underline{komutativnost} in \underline{asociativnost} obstaja ničla 0 in za vsak \(a \in K\) obstaja nasprotni element \((-a) \in K\), torej lahko vedno odštevamo.
        \[\forall a,b \in K: a + b = b + a\]
        \[\forall a,b,c \in K: (a + b) + c = a + (b + c)\]
        \[\forall a \in K \exists (-a) \in K: a + (-a) = a - a = 0\]
        
        \par
        Pri \underline{množenju} nimamo dodatnih zahtev, wazen uglašenosti s seštevanjem - \underline{distibutivnost}.
        \[\forall a,b,c \in K: a \cdot (b+c) = (a \cdot b) + (a \cdot c)\]

        Rečeno drugače, kolobar je grupa za seštevanje in zaprta za množenje.
    \end{definition}

    Če ima množenje kakšno dodatno lastnost, to lastnost običajno izpostavimo\
    \begin{center}
        \begin{tabular}{ c c c }
            Množenje je asociativno & \(\rightarrow\) & asociativni kolobar \\
            Množenje je komutativno & \(\rightarrow\) & komutativni kolobar \\
            Množenje ima enoto & \(\rightarrow\) & kolobar z enoto \\
            Vsak \(a \neq 0 \in K\) ima inverz \(a^{-1}\) & \(\rightarrow\) & kolobar z deljenjem
        \end{tabular}        
    \end{center}

    Literatura velikokrat v definiciji kolobarja zahteva tudi asociativnost in obstoj enote, zato bomo v nadaljevanju privzeli, da z izrazom "kolobar" mislimo na asociativen kolobar z enoto, komutativnost in deljenje pa bomo izrecno navedli.
    
    \begin{definition}[Obseg]
        Kolobarju, v katerem je množenje asociativno, komutativno in ima enoto ter ima operacijo deljenja, rečemo \textbf{obseg}.
    \end{definition}

    \subsection{Kolobarji}

    \begin{example}
        Če dani kolobar nima enote, mu jo lahko dodamo: \\
        \(K\) kolobar brez enote \\
        Če dodamo 1, smo prisiljeni dodati tudi \(-1,2,-2,3,-3,...\) \\
        Rešitev: na \(\mathbb{Z} \times K \) vpeljemo:
        \[
            (n,a)+(m,b) := (n+m,a+b)
        \]
        \[
            (n,a) \cdot (m,b) := (nm, nb + ma + ab)
        \]

        Ničla je \((0,0)\), nasprotni element je \(-(m,a) = (-m,-a)\), enota je \((1,0)\). Če je \(K\) komutativen oz. asociativen, je to tudi \(\mathbb{Z} \times K\).
    \end{example}

    Kaj pa, če imamo kolobar, v katerem nekateri elementi nimajo inverza, ampak bi jih želeli dodati? Poskusimo to storiti na takšen način, kot smo v osnovni šoli definirali racionalna števila s celimi števili, in sicer z uvedbo ulomkov. 

    \par
    Naj bo \(K\) kolobar z enoto. Za \(a,b \in K, b \neq 0\) vpeljemo simbole \(\frac{a}{b}\), s katerimi računamo 
    \[\frac{a}{b} + \frac{c}{d} := \frac{ad + bc}{bd}\] 
    \[\frac{a}{b} \cdot \frac{c}{d} := \frac{ac}{bd}\]

    Pojavi se težava: lahko se zgodi, da je \(bd = 0\), čeprav \(b \neq 0\) in \(d \neq 0\). Tega pri številih nismo vajeni vendar:
    \begin{example}[Matrike]
        \[
            \begin{bmatrix}
                0 & 1 \\
                0 & 0 \\
            \end{bmatrix}
            \cdot
            \begin{bmatrix}
                0 & 1 \\
                0 & 0 \\
            \end{bmatrix}
            =
            \begin{bmatrix}
                0 & 0 \\
                0 & 0 \\
            \end{bmatrix}
        \]
    \end{example}

    \begin{example}[\(\mathbb{Z}_{12}\): ostanki po modulu 12]
        \[2 \cdot 6 = 0\]
        \[3 \cdot 8 = 0\]
    \end{example}
    
    \begin{definition}[Delitelj niča]
        V kolobarju \(K\) je \(0 \neq a \in K\) \textbf{delitelj niča}, če obstaja tak \(b \neq 0\), da je \(a \cdot b = 0\).
    \end{definition}

    \begin{proposition}
        Delitelj niča nima inverza.
    \end{proposition}

    \begin{proof}
        Če za \(a \in K\) obstajata takšna neničelna \(b,c \in K\), da velja
        \begin{center}
            \(a \cdot b = 0\) in \(c \cdot a = 1\)
        \end{center}
        dobimo protislovje
        \[b = 1 \cdot b = (c\cdot a) \cdot b = c \cdot (a \cdot b) = c \cdot 0 = 0\]
    \end{proof}

    \begin{definition}[Celi kolobar]
        \textbf{Celi kolobar} je komutativni kolobar, v katerem ni deliteljev niča. Ekvivalentno, v celem kolobarju iz \(a \cdot b = 0\) sledi \(a = 0\) ali \(b = 0\).
    \end{definition}

    Naj bo \(K\) celi kolobar. Tvorimo \textbf{kolobar ulomkov} \(\overline{K}\): elementi so ulomki \(\frac{a}{b}\), vendar \(\frac{a}{b} \equiv \frac{c}{d}\), če je \(ad = bc\) (ulomke lahko krajšamo). Definiramo operaciji kot prej:
    \[\frac{a}{b} + \frac{c}{d} = \frac{ad + bc}{bd}\]
    \[\frac{a}{b} \cdot \frac{c}{d} = \frac{ac}{bd}\]

    Preverimo lahko, da je \(\overline{K}\) obseg.
    \par
    Zakaj moramo enačiti sorazmerne ulomke?
    \par
    Zaradi definicije operacij: \(\frac{a}{b} - \frac{ka}{kb} = \frac{kab - kab}{kb^2} = 0\), torej mora veljati \(\frac{ka}{kb} \equiv \frac{a}{b}\).

    \begin{remark}
        Z nekaj truda lahko vpeljemo ulomke tudi pri nekomutativnih kolobarjih in kolobarjih z deliterlji niča. Takrat dobimo kolobarje ulomkov, ki niso obsegi.
    \end{remark}

    Delitelj niča ne more biti obrnljiv, obrnljiv element pa ni delitelj niča. Ali je lahko element kolobarja niti obrnljiv niti delitelj niča?
    \begin{example}
        \(\mathbb{Z}\) nima deliteljev niča, obrnljiva pa sta le \(1\) in \(-1\).
    \end{example}

    \begin{example}
        \(\mathbb{Z}_{12}\)
        \begin{itemize}
            \item Delitelji niča: \(0,2,3,4,6,8,9,10\)
            \item Obrnljivi: \(1,5,7,11\)
        \end{itemize}
    \end{example}

    V končnih kolobarjih ni drugih možnosti, kar pove naslednji znameniti izrek.

    \begin{theorem}[Wedderburnov izrek]
        Končen kolobar brez deliteljev niča je obseg.
    \end{theorem}

    \begin{proof}
        Naj bo \(K\) končen kolobar brez deliteljev niča. Dokazati moramo, da so vsi elementi \(K - \{0\}\) obrnljivi.
        \par
        Za poljuben \(a \in K - \{0\}\) definiramo funkcijo
        \[l_a: K \rightarrow K\] 
        \[l_a(x) := a \cdot x\]
        \par
        Recimo, da za neka \(x,y \in K\) velja \(l_a(x) = l_a(y)\). Potem je \(ax = ay\), torej \(a(x-y) = 0\). Ker \(K\) nima deliteljev niča, sledi \(x = y\). Torej je \(l_a\) injektivna in ker slika iz \(K\) v \(K\), tudi bijektivna.
        \par
        Ker je \(l_a(K) = K\), je \(l_a(b) = a \cdot b = 1\) za nek \(b \in K\). Ponovimo razmislek za \(d_a (x):= x \cdot a\) in dobimo, da je \(d_a(c) = a \cdot c = 1\) za nek \(c \in K\). Iz \(c = c \cdot 1 = c \cdot (a \cdot b) = (c \cdot a) \cdot b = 1 \cdot b = b\) sledi \(c = b = a^{-1}\). \(K\) je torej kolobar z deljenjem. Če je \(K\) komutativen, je obseg.
        \par
        Dokaz komutativnosti zahteva nekaj dodatne teorije grup, zato ga ne bomo natančno navedli. Ideja je, da je center \(Z(K)\) (tj. elementi \(K\), ki komutirajo z vsemi elementi \(K\)) obseg, celoten \(K\) pa je vektorski prostor nad \(Z(K)\). Iz primerjave multiplikativnih grup lahko izpeljemo, da je \(K\) 1-razsežen vektorski prostor nad \(Z(K)\), torej \(K = Z(K)\).
    \end{proof}

    \begin{corollary}
        \(\mathbb{Z}_n\) je obseg \(\Leftrightarrow\) \(n\) je praštevilo.
    \end{corollary}

    \begin{definition}[Karakteristika kolobarja]
        \textbf{Karakteristika} kolobarja \(K\) je najmanjši \(n \in \mathbb{N}\), za katerega velja, da je \(n \cdot a = a+...+a=0\). Zapišemo jo kot \(char(K)=n\) ali \(char K = n\). Če tak \(n\) ne obstaja, potem pišemo \(char(K) = 0\).
    \end{definition}

    \begin{proposition}
        Naj bo \(K\) kolobar.
        \begin{enumerate}[label=\alph*)]
            \item Če \(1 \in K\), potem je \(char(K) = \) red enote \(= min\ n\), da je \(n \cdot 1 = 0\).
            \item Če \(K\) nima deliteljev niča, potem je \(char(K)\) praštevilo ali \(0\). 
        \end{enumerate}
    \end{proposition}

    \begin{proof}[Dokaz \emph{(a)}]
        Naj bo \(n\) red enote, torej je \(n \cdot 1 = 0\). Potem za vsak \(a \in K\) velja:
        \[n \cdot a = (n \cdot 1) \cdot a = 0 \cdot a = 0\]
        Iz tega sledi, da je \(char K \le n\). Ker je red \(1\) enak \(n\), je \(char K \ge n\), torej \(char K = n\).
    \end{proof}

    \begin{proof}[Dokaz \emph{(b)}]
            Denimo, da je \(char K = k \cdot l\) za \(k,l > 1\). Potem je \(0 = k \cdot l \cdot 1 = k \cdot l\). Ker v K ni deliteljev niča, je \(k \cdot 1 = 0\) ali \(l \cdot 1 = 0\), zato je po \((a)\) \(char K < k \cdot l\). Protislovje.
    \end{proof}

    \begin{definition}
        Naj bosta \(K,L\) kolobarja.
        \par
        \(f\): \(K \rightarrow L\) je \textbf{homomorfizem kolobarjev}, če velja:
        \begin{center}
            \(f(a + b) = f(a) + f(b)\)
        \end{center}
        \begin{center}
            \(f(a \cdot b) = f(a) \cdot f(b)\)
        \end{center}
        za poljubna \(a,b \in K\).
    \end{definition}

    Bijektivnemu homomorfizmu rečemo \textbf{izomorfizem}.
    \par
    Poglejmo nekaj primerov homomorfizmov:

    \begin{example}
        \(\mathbb{Z} \xrightarrow{mod\ n} \mathbb{Z}_n\) je homomorfizem (dokaz ponovi korake dokaza, da je \(\mathbb{Z}_n\) kolobar).
    \end{example}

    \begin{example}
        Konjugiranje \(\mathbb{C} \to \mathbb{C}\), \(z \mapsto \overline{z}\) je izomorfizem kolobarjev.
    \end{example}

    \begin{example}
        Za poljuben \(a \in \mathbb{R}\) definiramo \(f_a\): \(\mathbb{Z}[x] \rightarrow \mathbb{R}\) s predpisom \(f_a(p) = p(a)\)
        \[f_a(p + q) = (p + q)(a) = p(a) + q(a) = f_a(p) + f_a(q)\]
        \[f_a(p + q) = (p \cdot q)(a) = p(a) \cdot q(a) = f_a(p) \cdot f_a(q)\]
    \end{example}

    \begin{example}
        \(a \mapsto 
        \begin{bmatrix} 
            a & 0 \\ 
            0 & 0 
        \end{bmatrix}\) 
        je homomorfizem \(\mathbb{R} \rightarrow M_2(\mathbb{R})\).
    \end{example}

    \begin{example}
        Splošneje \(f_a\): \(\mathscr{C}(\mathbb{R}, \mathbb{R}) \rightarrow \mathbb{R}\) s predpisom \(f_a(g) := g(a)\) je tudi homomorfizem kolobarjev.
    \end{example}

    \begin{example}
        \(a \mapsto 
            \begin{bmatrix} 
                0 & a \\ 
                0 & 0 
            \end{bmatrix}\) 
        \underline{ni} homomorfizem \(\mathbb{R} \rightarrow M_2(\mathbb{R})\) (ohranja seštevanje, ne pa množenja).
    \end{example}

    \begin{example}
        Preslikava \(\mathbb{C} \rightarrow M_2(\mathbb{R})\) s predpisom \(a + bi \mapsto 
            \begin{bmatrix}
                a & b \\
                -b & a
            \end{bmatrix}\)
        je homomorfizem.
    \end{example}

    \begin{example}
        \(a \mapsto a^p\) je homomorfizem \(\mathbb{Z}_p \rightarrow \mathbb{Z}_p\) (\(p\) praštevilo).
    \end{example}

    \begin{proposition}
        Za homomorfizem \(f\): \(K \rightarrow L\) velja:
        \begin{enumerate}[label=\alph*)]
            \item \(f(0) = 0\)
            \item \(f(-a) = -f(a)\)
        \end{enumerate}
    \end{proposition}
    
    \begin{proof}[Dokaz \emph{(a)}]
        \(f(0) = f(0 + 0) = f(0) + f(0)\)
    \end{proof}

    \begin{proof}[Dokaz \emph{(b)}]
        \(0 = f(0) = f(a+(-a)) = f(a) + f(-a)\)
    \end{proof}

    \begin{definition}[Unitalen homomorfizem]
        Homomorfizem, za katerega velja \(f(1) = 1\), je \textbf{unitalen}.
    \end{definition}

    \begin{proposition}
        Če je \(f\) unitalen homomorfizem in \(a\) obrnljiv, je \(f(a)\) obrnljiv.
    \end{proposition}

    \begin{proof}
        Naj bo \(f: K \to L\) unitalen homomorfizem kolobarjev in \(a \in K\) obrnljiv. Potem je
        \[1 = f(1) = f(a \cdot a^{-1}) = f(a) \cdot f(a^{-1})\]
        Potem je \(f(a^{-1})\) desni inverz za \(f(a)\). Po podobnem razmisleku vidimo, da je \(f(a^{-1})\) tudi levi inverz za \(f(a)\). Torej je \(f(a^{-1}) = f(a)^{-1}\).
    \end{proof}

    \begin{proposition}
        Naj bo \(f\): \(K \to L\) homomorfizem kolobarjev. Potem je \(Im f\) podkolobar v \(L\).
    \end{proposition}

    \begin{proof}
        Vzamemo poljubna \(x,y \in Im f\). Potem obstajata takšna \(a,b \in K\), da je \(f(a) = x\) in \(f(b) = y\) in zato velja
        \[x + y = f(a) + f(b) = f(a + b) \in Im f\]
        \[x \cdot y = f(a) \cdot f(b) = f(a \cdot b) \in Im f\]
        Torej je \(Im f\) podkolobar v \(L\).
    \end{proof}

    \begin{proposition}
        Naj bo \(f: K \to L\) homomorfizem kolobarjev. Potem je \(Ker f\) podkolobar v \(K\).
    \end{proposition}

    \begin{proof}
        Vzamemo poljubna \(x,y \in Ker f\). Potem je \(f(x + y) = f(x) + f(y) = 0 + 0 = 0\), torej je \(x + y \in Ker f\). Podobno je \(f(x \cdot y) = f(x) \cdot f(y) = 0 \cdot 0 = 0\), torej je \(x \cdot y \in Ker f\). 
    \end{proof}


    \begin{definition}[Ideal kolobarja]
        Podkolobar \(I \leq K\) je \textbf{ideal} v \(K\), če za vsak \(a \in K, x \in I\) velja \(a \cdot x \in I\) (levi ideal) in \(x \cdot a \in I\) (desni ideal). Označimo ga z \(I \triangleleft K\).
    \end{definition}

    \begin{example}
        \(\{0\} \triangleleft K\) in \(K \triangleleft K\) sta "neprava" ideala.
    \end{example}

    \begin{example}
        \(n \mathbb{Z} \triangleleft \mathbb{Z}\)
    \end{example}

    \begin{example}
        \(\{a_1 x + a_2 x^2 + a_3 x^3 + ... + a_n x^n\ |\ a \in \mathbb{R}\} \triangleleft \mathbb{R}[x]\) so polinomi, za katere je \(p(0) = 0\). Splošneje, za \(a \in K\) je \(\{p \in K[x]\ |\ p(a) = 0\} \triangleleft K[x]\).
    \end{example}

    \begin{example}
        \(\{
            \begin{bmatrix}
                x & 0 \\
                y & 0 \\
            \end{bmatrix}\
            |\ x,y \in \mathbb{R}
        \} \subseteq M_2(\mathbb{R})\) je podkolobar.

        \[
            \begin{bmatrix}
                x & 0 \\
                y & 0 \\
            \end{bmatrix}
            \pm
            \begin{bmatrix}
                z & 0 \\
                w & 0 \\
            \end{bmatrix}
            =
            \begin{bmatrix}
                x \pm z & 0 \\
                y \pm w & 0 \\
            \end{bmatrix}
        \]

        \[
            \begin{bmatrix}
                x & 0 \\
                y & 0 \\
            \end{bmatrix}
            \cdot
            \begin{bmatrix}
                z & 0 \\
                w & 0 \\
            \end{bmatrix}
            =
            \begin{bmatrix}
                xz & 0 \\
                yw & 0 \\
            \end{bmatrix}
        \]
        Poleg tega je
        \[
            \begin{bmatrix}
                a & b \\
                c & d \\
            \end{bmatrix}
            \cdot
            \begin{bmatrix}
                x & 0 \\
                y & 0 \\
            \end{bmatrix}
            =
            \begin{bmatrix}
                ax + by & 0 \\
                cx + dy & 0 \\
            \end{bmatrix}
        \]
        \[
            \begin{bmatrix}
                x & 0 \\
                y & 0 \\
            \end{bmatrix}
            \cdot
            \begin{bmatrix}
                a & b \\
                c & d \\
            \end{bmatrix}
            =
            \begin{bmatrix}
                ax & bx \\
                ay & by \\
            \end{bmatrix}
        \]
        Podana množica je levi ideal. 
    \end{example}

    \begin{proposition}
        Naj bo \(f\): \(K \to L\) homomorfizem kolobarjev. Potem je \(Ker f\) (dvostranski) ideal v \(K\).
    \end{proposition}

    \begin{proof}
        Vzemimo \(x \in Ker f\) in \(a \in K\). \(f(x) \cdot f(a) = f(x) \cdot f(a) = 0 \cdot f(a) = 0\), torej je \(x \cdot a \in Ker f\). Analogno, \(f(a \cdot x) = f(a) \cdot f(x) = f(a) \cdot 0 = 0\), torej je tudi \(a \cdot x \in Ker f\).
    \end{proof}

    Za \(x \in K\) je \(K \cdot x = \{a \cdot x\ |\ a \in K\}\) levi ideal v \(K\), \(x \cdot K\) pa desni. Dvostranski ideal, ki ga generira \(x\), pa ni \(K \cdot x \cdot K\), ker ni zaprt za seštevanje. Potrebno je vzeti vse možne vsote izrazov \(a \cdot x \cdot b\).

    \begin{definition}[Glavni ideal]
        Ideal, ki ga generira en sam element, imenujemo \textbf{glavni ideal} in ga zapišemo z \((x)\), kjer je \(x \in K\).
    \end{definition}

    \begin{proposition}
        V \(\mathbb{Z}\) so vsi ideali glavni.
    \end{proposition}

    \begin{proof}
        Naj bo \(I \triangleleft \mathbb{Z}\). Če je \(I = \{0\}\), potem je \(I = (0)\). Recimo, da \(I \neq (0)\). Potem obstaja vsaj eno pozitivno število v \(I\). Vzemimo najmanjše pozitivno število \(a \in I\). Pokazati moramo, da \(a\) deli vse elemente \(I\).
        \par
        Poljuben \(b \in I\) je po izreku o deljenju enak \(b = k \cdot a + r\) za natanko določena \(k \in \mathbb{Z}\) in \(0 \le r < a\). Potem je \(r = b - k \cdot a \in I\). Ker je \(a\) najmanjše pozitivno število v \(I\), je \(r = 0\), torej je \(b = k \cdot a\). \(I\) je torej generiran z \(a\).
    \end{proof}

    \begin{definition}[Glavnoidealski kolobar]
        Kolobar, v katerem so vsi ideali glavni, je \textbf{glavnoidealski}.
    \end{definition}

    Naj bo \(I \triangleleft K\) dvostranski ideal. Na \(K\) vpeljemo relacijo \(\sim\):
    \[a \sim b \Longleftrightarrow a - b \in I\]

    \begin{proposition}
        \(\sim\) je ekvivalenčna relacija.
    \end{proposition}

    \begin{proof}
        \(\)\par
        \(a - a = 0 \in I\), torej je \(a \sim a\).
        \par
        Če \(a - b \in I\), je tudi \(b - a \in I\), torej \(a \sim b \implies b \sim a\).
        \par
        Če \(a - b \in I\) in \(b - c \in I\), je tudi \(a - c \in I\). Torej je \(a \sim b, b\ \sim c \implies a \sim c\).
    \end{proof}

    \begin{definition}[Kvocientni kolobar]
        Množico ekvivalenčnih razredov relacije \(\sim\) označimo z \(K / I\) in jo imenujemo \textbf{kvocientni kolobar}. 
    \end{definition}

    Ekvivalenčni razred elementa \(a \in K\) označimo \([a]\) ali \(a + I\) (oznaka je smiselna, ker je \([a] = \{a + x\ |\ x \in I\}) \equiv a + I\).

    \par
    Elemente \(K / I\) naravno seštevamo in množimo:
    \[(a+I) + (b+I) := (a+b) + I\]
    \[(a+I) \cdot (b+I) := (a \cdot b) + I)\]

    Da bosta operaciji dobro definirani, moramo preveriti, sta neodvisni od izbire predstavnikov ekvivalečnih razredov. Najprej preverimo operacijo seštevanja:
    \par
    Če \(a' + I = a + I\) in \(b' + I = b + I\), je \(a - a' \in I\) in \(b - b' \in I\).
    \[(a - a') + (b - b') = (a + b) - (a' + b') = (a + b) + I = (a' + b') + I\]
    
    Preverimo še za množenje:
    \begin{center}
        \(a' = a + x\) in \(b' = b + y\) za \(x,y \in I\).
    \end{center}
    
    \[a' \cdot b' = a \cdot b + ay + xb + xy\]
    \[(a' \cdot b') + I = (a \cdot b) + I\]

    Zdaj vemo, da sta operaciji dobro definirani. Sedaj preverimo, da je \(K / I\) res kolobar.

    \begin{proposition}
        Naj bo \(K\) kolobar in \(I \triangleleft K\). Potem je \(K / I\) tudi kolobar.
    \end{proposition}

    \begin{proof}
        Vzemimo poljubne \(a,b,c \in K / I\). Najprej preverimo, da je grupa za seštevanje:
        \begin{center}
            asociativnost: \(((a+b) + c) + I = (a+I) + (b+I) + (c+I) = (a+I) + ((b+c) + I) = (a + (b+c)) + I\) \\
            komutativnost: \((a+b) + I = (a+I) + (b+I) = (b+I) + (a+I) = (b+a) + I\) \\
            negativni element: \((a+b) + I = 0 + I \implies a + I = -b + I\)
        \end{center} 

        Preverimo še asociativnost in obstoj enote za množenje:
        \begin{center}
            asociativnost: \(((a \cdot b) \cdot c) + I = (a+I) \cdot (b+I) \cdot (c+I) = (a+I) \cdot ((b \cdot c) + I) = (a \cdot (b \cdot c)) + I\) \\
            komutativnost: \((a \cdot b) + I = (a+I) \cdot (b+I) = (b+I) \cdot (a+I) = (b \cdot a) + I\) \\
        \end{center}
    \end{proof}

    \begin{theorem}[Izrek o izomorfizmu]
        Naj bo \(f\): \(K \to L\) homomorfizem kolobarjev. Potem je \(Ker f \triangleleft K\) in imamo naravni izomorfizem:
        \begin{center}
            \(\overline{f}\): \(K / Ker f \to Im f\) s predpisom \(\overline{f}(x + Ker f) := f(x)\)
        \end{center}
    \end{theorem}

    \begin{proof}
        \(Ker f \triangleleft K\) že vemo.
        \par
        Za \(u \in Ker f\) je \(f(x+u)=f(x)\), zato je definicija \(\overline{f}\) neodvisna od predstavnika razreda.
        \par
        \(\overline{f}\) je aditivna:
        \[\overline{f}((x + Ker f) + (y + Ker f)) = \overline{f}((x+y) + Ker f)\] \[= f(x+y) = f(x) + f(y) = \overline{f}(x + Ker f) + \overline{f}(y + Ker f)\]

        Za množenje opravimo analogni razmislek.

        \[Ker \overline{f} = \{x + Ker f\ |\ \overline{f}(x + Ker f) = 0 \} = \{x\ |\ f(x) = 0\} = \{0 + Ker f\}\]
        \(\overline{f}\) je torej injektivna.

        \[Im \overline{f} = Im f\]
        \(\overline{f}\) je surjektivna.
    \end{proof}

    \begin{proposition}
        Komutativen kolobar \(K\) je obseg natanko tedaj, ko nima pravih idealov (tj. edina ideala sta \((0)\) in \(K\)).
    \end{proposition}

    \begin{proof}[Dokaz \(\implies\)]
        Recimo, da je \(K\) obseg in \(I \triangleleft K\) neničelni ideal v \(K\) (\(I \neq (0)\)). Potem obstaja \(0 \neq x \in I\). Ker je \(x\) obrnljiv v \(K\), potem za vsak \(a \in K\) velja \(a = (a \cdot x^{-1}) \cdot x \in I\), torej je \(I = K\).
    \end{proof}

    \begin{proof}[Dokaz \(\Longleftarrow\)]
        Recimo, da \(K\) nima pravih idealov. Naj bo \(0 \neq a \in K\). Potem je \((a) = K \cdot a = a \cdot K \triangleleft K\). Ker v \(K\) ni pravih idealov, je \((a) = K\). Ker  \(K\) vsebuje enoto in je celoten generiran z \(a\), je \(a \cdot b = b \cdot a = 1\) za nek \(b \in K\). Sledi, da je \(a\) obrnljiv, in ker smo za \(a\) izbrali poljuben element \(K\), so vsi elementi v \(K\) obrnljivi. Torej je \(K\) obseg.
    \end{proof}

    Poiščimo ideale v \(K / I\). Za pomoč definirajmo funkcijo
    \begin{center}
        \(q\): \(K \to K/I\) \\
        \(x \mapsto x+I\) \\
    \end{center}

    \begin{proposition}
        Naj bo \(K\) komutativen kolobar in \(I \triangleleft K\).
        \begin{enumerate}[label=\alph*)]
            \item Če je \(J \triangleleft K\), potem je \(q(J) \triangleleft K / I\).
            \item Če je \(J \triangleleft K / I\), potem je \(q^{-1}(J) \triangleleft K\) in \(I \triangleleft q^{-1}(J)\) 
        \end{enumerate}
    \end{proposition}

    \begin{proof}[Dokaz \emph{(a)}]
        Vzamemo \(a \in K\) in \(x \in J\). Potem je \(a+I \in K / I\) in \(x + I \in q(J)\).
        \[(a+I)(x+I) = (a \cdot x) + I \in q(J)\]
    \end{proof}

    \begin{proof}[Dokaz \emph{(b)}]
        Vzamemo \(a \in K\) in \(x \in q^{-1}(J)\). Potem je \(a+I \in K / I\) in \(x + I \in J\).
        \[q(a \cdot x) = (a \cdot x) + I = (a+I)(x+I) \in J \implies ax \in q^{-1}(J)\]
    \end{proof}

    Pokazali smo, da so ideali v \(K / I\) natanko ideali oblike \(J / I\). \(J\) je torej natanko ideal v \(K\), ki vsebuje \(I\), hkrati pa je \(I \triangleleft J\).

    \begin{definition}[Maksimalni ideal]
        \textbf{Maksimalni ideal} v kolobarju \(K\) je pravi ideal, ki ni vsebovan v nobenem drugem pravem idealu.
    \end{definition}

    \begin{theorem}
        Naj bo \(K\) kolobar in \(I \triangleleft K\). \(K / I\) je obseg natanko takrat, ko je \(I\) maksimalni ideal v \(K\).
    \end{theorem}

    \begin{proof}
        Vemo, da je \(K / I\) obseg natanko takrat, ko nima pravih idealov. Ker nima pravih idealov, ni idealov v \(K\), ki bi vsebovali \(I\), torej je \(I\) maksimalen ideal.
    \end{proof}

    % \begin{theorem}
    %     Naj bo \(R\) obseg. \(I \triangleleft R[x]\) je maksimalen natanko tedaj, ko je \(I = (p(x))\) za nek nerazcepen polinom \(p(x)\).
    % \end{theorem}

    % \begin{proof}[Dokaz \(\implies\)]
    %   Vzamemo \(p(x) = q(x) \cdot r(x) \in R[x]\), kjer je \(q\) nekonstanten nerazcepen polinom in \(r\) neničeln polinom. Ker je \(st p \ge st q\), je \((p) \triangleleft (q) \triangleleft R[x]\). 
    % \end{proof}

    % \begin{proof}[Dokaz \(\Longleftarrow\)]
    %     Naj bo \(I = (p)\) za nek nerazcepen polinom \(p \in R[x]\). Če ni maksimalen, je \((p) \triangleleft J \triangleleft R[x]\)
    % \end{proof}


    \pagebreak
    \subsection{Obsegi}

    Obseg je komutativen kolobar, v katerem so vsi neničelni elementi obrnljivi. Stvari, ki nas zanimajo pri kolobarjih, npr. ulomki, ideali, kvocienti itd., so pri obsegih precej nezanimive. Namesto tega se bomo ukvarjali z bolj zanimivimi pojmi.

    \begin{definition}[Razširitev obsega]
        Če je \(K\) podobseg obsega \(F\), pravimo, da je \(F\) \textbf{razširitev} obsega \(K\) in pišemo \(K \le F\).
    \end{definition}

    \begin{proposition}
        če je \(F\) razširitev obsega \(K\), je \(F\) vektorski prostor nad \(K\).
    \end{proposition}

    Dimenzijo \(K\)-vektorskega prostora \(F\) \(dim_{K}F\) običajno označimo z \([F:K]\). Če je dimenzija končna, pravimo, da je \(F\) \textbf{končne razširitev}, sicer pa je \textbf{neskončna razširitev} \(K\).

    \begin{theorem}
        Za obsege \(K \le F \le E\) velja \([E:K] = [E:F] \cdot [F:K]\).
    \end{theorem}

    \begin{proof}
        Za neskončne razširitve je očitno, zato se omejimo na končne razširitve.
        \par
        Naj bo \(x_1,...,x_m\) baza za \(F\) nad \(K\) in naj bo \(y_1,...,y_n\) baza za \(E\) nad \(F\). Pokazati moramo, da je \(\{x_i y_j\ |\ i = 1,...,m, j = 1,...,n\}\) baza \(E\) nad \(K\).
        \par
        Vzemimo \(e \in E\). \(e = f_1 y_1 + ... + f_n y_n\) za \(f_1,...,f_n \in F\). Obenem je \(f_i = k_{i1} x_1 + ... + k_{im} x_m\) za \(k_{ij} \in K\). Torej je \(e = \sum_{i = 1}^n \sum_{j = 1}^m k_{ij} x_j y_i\).
        \par
        Pokažimo tudi linearno neodvisnost baze. Recimo, da je linearno odvisna. Potem je \(0 = \sum_{i = 1}^n \sum_{j = 1}^m k_{ij} x_j y_i\). Ker so \(y_i\) linearno neodvisni, ke \(\sum_{j = 1}^m = 0\). Ker so tudi \(x_j\) linearno neodvisni, je \(k_{ij} = 0\) za vse \(i\) in \(j\). 
    \end{proof}

    Poglejmo si, kako izgleda najmanjša razširitev obsega \(K\), ki vsebuje nek \(a \in F\).
    
    \begin{proposition}
        \(\)\par
        \begin{enumerate}[label=\alph*)]
            \item Najmanjši podkolobar \(F\), ki vsebuje \(K\) in \(a \in F\), je \(K[a] = \{p(a)\ |\ p \in K[x]\}\).
            \item Najmanjši podobseg \(F\), ki vsebuje \(K\) in \(a \in F\), je \(K(a) = \{\frac{p(a)}{q(a)}\ |\ p,q \in K[x], q(a) \neq 0\}\).
        \end{enumerate}
    \end{proposition}

    \begin{proof}[Dokaz \emph{(a)}]
        Vsak kolobar, ki vsebuje \(K\) in \(a\), mora vsebovati tudi vse potence \(a\) in njihove \(K\)-linearne kombinacije \(k_0 + k_1 a + ... + k_n a^n\). Nadaljevanje dokaza izpuščeno.
    \end{proof}

    \begin{proof}[Dokaz \emph{(b)}]
        Obseg mora poleg \(K\)-linearnih kombinacij potenc \(a\) vsebovati še vse kvociente, katere predstavimo z ulomki oblike \(\frac{p(a)}{q(a)}\). Konstruiramo obseg ulomkov.
    \end{proof}

    Kolobar lahko razširimo z več elementi \(a_1,a_2,..\) naenkrat. Takšne razširitve označimo z \(K[a_1,a_2,...]\) za kolobarje in \(K(a_1,a_2,...)\) za obsege. Vrstni red dodanih elementov je lahko v zapisu poljuben.

    \begin{definition}[Enostavna razširitev]
        Razširitev obsega \(K\) je \textbf{enostavna}, če smo obsegu \(K\) dodali eden element \(a \in F\).
    \end{definition}

    Definiramo poseben homomorfizem kolobarjev:
    \begin{center}
        \(\phi_a\): \(K[x] \to F\) \\
        \(\phi_a\): \(p(x) \mapsto p(a)\)
    \end{center}

    Zanima nas, ali je \(\phi_a\) injektiven.
    \par
    Če je \(\phi_a\) injektiven, je \(Ker f = \{0\}\), torej \(a\) ni ničla nobenega (trivialnega) polinoma s koeficienti v \(K\). Tedaj pravimo, da je \(a\) \textbf{transcendenten} nad \(K\).
    \par
    Če pa \(\phi_a\) ni injektiven, potem obstaja netrivialen polinom s koeficienti v \(K\), v katerem je \(a\) ničla. Tedaj pravimo, da je \(a\) \textbf{algebraičen} nad \(K\).
    
    \begin{definition}[Algebraična razširitev]
        Naj bo \(F\) razširitev nad obsegom \(K\). Če so vsi elementi \(F\) algebraični nad \(K\), pravimo, da je \(F\) \textbf{algebraična razširitev} nad \(K\).
    \end{definition}

    \begin{definition}
        Naj bo \(F\) razširitev nad obsegom \(K\). Če je vsaj eden element \(F\) transcendenten nad \(K\), je \(F\) \textbf{transcendentna razširitev} nad \(K\). 
    \end{definition}

    \begin{remark}
        Za dano število je običajno zelo težko dokazati, da je transcendentna nad \(\mathbb{Q}\). Na primer, transcendentnost \(e\) je bila dokazana leta 1873, transcendentnost \(\pi\) je bila dokazana leta 1882, še vedno pa ne vemo, ali je \(\pi + e\) transcendentna nad \(\mathbb{Q}\).
    \end{remark}

    \begin{theorem}
        če je \(a\) transcendenten nad \(K \le F\), potem je \(K[a] \cong K[x]\) in \(K(a) \cong K(x)\).
    \end{theorem}
\end{document}